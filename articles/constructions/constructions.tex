\documentclass[11pt,paper=letter]{scrartcl}
\usepackage[noextlink]{cjquines}

\begin{document}

\title{Constructions}
\author{Carl Joshua Quines}
\date{November 5, 2018}

\maketitle

\begin{abstract}

\noindent\textbf{Disclaimer:} This is an \emph{experimental} problem set. I make no guarantees of usefulness. Take all of this with a grain of salt.

All of the problems in this set have solutions that make use of a construction that makes it easier. The idea is ``certain parts of a problem should signal trying a construction''. It may not solve the problem, but is often a key step. The problems are categorized under type of construction, and what feature the problems have that triggers the construction.

Hints are provided after the problems. References used in preparing this problem set appear at the end.

\end{abstract}

\section{Problems}

\subsection{Unrolling}

If you have a condition like $AB + CD = EF$, you \emph{almost always} want to ``unroll'' or ``open the gates''.\footnote{Problem 5 in section Grab Bag is one of the few exceptions I know.} These conditions have a point in common on the same side or both sides. Here, we distinguish between two kinds, based on the construction:

\subsubsection*{Same point on the same side}

The first is ``constructive'': turning the sum of two segments to a segment. If $AX + AY$ appears on one side, construct a point $Y'$ on ray $AX$ such that $XY' = AY$. Then $AX + AY = AY'$, and $AY'$ hopefully makes an isosceles triangle, parallelogram, isosceles trapezoid, or cyclic quadrilateral.

Note that you can try constructing on ray $AY$ instead. Trying to construct on the opposite direction seems to help sometimes.

\begin{enumerate}
  \item (\href{https://aops.com/community/c6h1513473}{IGO 2017/E3}) In regular pentagon $ABCDE$, point $F$ lies on segment $AB$ such that $\angle FCD = 90^{\circ}$. Prove that $AE + AF = BE$.

  \item (\href{https://aops.com/community/c6h116955}{Brazil 2006/1}) In triangle $ABC$, the internal bisector of $\angle B$ meets $AC$ at $P$. Let $I$ be the incenter of triangle $ABC$. Prove that if $AP + AB = CB$, then $API$ is an isosceles triangle.

  \item (\href{https://aops.com/community/c6h58299}{IMO 1997/2}) Angle $A$ is the smallest in triangle $ABC$. The points $B$ and $C$ divide the circumcircle of the triangle into two arcs. Let $U$ be an interior point of the arc between $B$ and $C$ which does not contain $A$. The perpendicular bisectors of $AB$ and $AC$ meet line $AU$ at $V$ and $W$, respectively. Lines $BV$ and $CW$ intersect at $T$. Show that $AU = TB + TC$.

  \item (van Schooten's theorem) Let $P$ be a point on minor arc $BC$ of equilateral triangle $ABC$. Then $PA = PB + PC$.

  \item (\href{https://aops.com/community/c132h1546692}{MPfG Olympiad 2017/3}) Let $ABCD$ be a cyclic quadrilateral such that $\angle BAD \le \angle ADC$. Prove that $AC + CD \le AB + BD$.

  \item (\href{https://aops.com/community/c6h17469}{IMO 2001/5}) Let $ABC$ be a triangle with $\angle BAC = 60^{\circ}$. Let $AP$ bisect $\angle BAC$ and let $BQ$ bisect $\angle ABC$, with $P$ on $BC$ and $Q$ on $AC$. If $AB + BP = AQ + QB$, what are the angles of the triangle?
\end{enumerate}

\subsubsection*{Same point on opposite sides}

The other is ``destructive'': turning the difference of two segments to a segment. If $AX$ appears on one side and $AY$ on the other, choose a point $X'$ on ray $AY$ such that $AX' = AX$. This makes $AXX'$ an isosceles triangle, and $X'Y$ the difference, where $X'Y$ hopefully makes an isosceles triangle, parallelogram, isosceles trapezoid, or cyclic quadrilateral.

Note that you can try constructing on ray $AX$ instead. Trying to construct on the opposite direction seems to help sometimes.

A specific form of this looks like $AX + BY = XY$. In this case, there's a point $P$ on $XY$ that forms isosceles triangles $AXP$ and $BYP$. This may or may not help: sometimes the point on rays $AX$ or $BY$ helps more.

\begin{enumerate}
  \item (IGO 2016/E5) Let $ABCD$ be a convex quadrilateral such that $\angle ADC = 135\dg$ and $\angle ADB - \angle ABD = 2\angle DAB = 4\angle CBD$. If $BC = CD\sqrt2$, prove that $AB = BC + AD$.

  \item (\href{https://aops.com/community/c6h60782}{IMO 1985/1}) A circle has center on the side $AB$ of the cyclic quadrilateral $ABCD$. The other three sides are tangent to the circle. Prove that $AD+BC=AB$.

  \item (\href{https://aops.com/community/c6h1525185}{JBMO SL 2014/G1}) Let $ABC$ be a triangle with $\angle ABC = \angle BCA = 40^{\circ}$. The angle bisector of $\angle ABC$ meets side $AC$ at $D$. Prove that $BD + DA = BC$.

  \item (\href{https://aops.com/community/c6h406786}{Bosnia and Herzegovina TST 2011/1}) Let $ABC$ be a triangle such that $AB + AC = 2BC$. Show that the midpoints $M$ of $AB$, $N$ of $AC$, its incenter $I$, and $A$ lie on the same circle.

  \item (\href{https://aops.com/community/c6h1632776}{Polish JMO Finals 2018/2}) Let $ABCD$ be a trapezoid with $AB$ parallel to $CD$ and $AB + CD = AD$. Diagonals $AC$ and $BD$ intersect at $E$. A line passing through $E$ parallel to $AB$ cuts $AD$ at $F$. Prove that $\angle BFC = 90^{\circ}$. 
% Take P on AD such that AB = AP, CD = PD.

  \item (\href{https://aops.com/community/c6h365230}{IZhO 2010/2}) In a cyclic quadrilateral $ABCD$ with $AB=AD$ points $M$ and $N$ lie on the sides $BC$ and $CD$ respectively so that $MN=BM+DN$. Lines $AM$ and $AN$ meet the circumcircle of $ABCD$ again at points $P$ and $Q$ respectively. Prove that the orthocenter of the triangle $APQ$ lies on the segment $MN$.

  \item (\href{https://aops.com/community/c6h418635}{ISL 2010/G5}) Let $ABCDE$ be a convex pentagon such that $BC \parallel AE$, $AB = BC + AE$, and $\angle ABC = \angle CDE.$ Let $M$ be the midpoint of $CE,$ and let $O$ be the circumcenter of triangle $BCD.$ Given that $\angle DMO = 90^{\circ},$ prove that $2 \angle BDA = \angle CDE$. 
  % Choose point T on ray AE such that AT = AB
  % Reflect $B,D$ about $M$ to get $L,K$ respectively.
  % Now observe that $OK=OD=OB=OC$ implies that $B,K,D,C$ are concyclic. Now we have by the length conditions, $A,E,L$ are collinear and $AB=AL$. => BALD => BA = AL
\end{enumerate}

\subsection{Parallelograms}

\subsubsection*{Existing parallelogram or isogonality}

If there is an existing parallelogram, you should try constructing another if it's useful. In particular, if you have parallelogram $ABCD$, and a point $E$, then the same point $F$ completes parallelograms $ABEF$ and $CDFE$. 

Completing this diagram is helpful when you have other information about the angles: here, completing the diagram forms isogonal points or cyclic quadrilaterals because of the additional information. 

\begin{enumerate}
  \item (\href{http://www.aops.com/community/c6h518987}{BMO2 2013/2}) The point $P$ lies inside triangle $ABC$ so that $\angle ABP = \angle PCA$. The point $Q$ is such that $PBQC$ is a parallelogram. Prove that $\angle QAB = \angle CAP$.

  \item (\href{https://aops.com/community/c6h405503}{Canada 1997/4}) The point $O$ is situated inside the parallelogram $ABCD$ such that $\angle AOB + \angle COD = 180\dg$. Prove that $\angle OBC = \angle ODC$. % parallelograms beget parallelograms because of angles? see note above isogonality lemma

  \item (\href{https://aops.com/community/c6h546176}{ISL 2012/G2}) Let $ABCD$ be a cyclic quadrilateral whose diagonals $AC$ and $BD$ meet at $E$. The extensions of the sides $AD$ and $BC$ beyond $A$ and $B$ meet at $F$. Let $G$ be the point such that $ECGD$ is a parallelogram, and let $H$ be the image of $E$ under reflection in $AD$. Prove that $D, H, F$, and $G$ are concyclic.

  \item (\href{http://www.aops.com/community/c6h598540}{Taiwan TST2 2014/6}) Let $P$ be a point inside triangle $ABC$, and suppose lines $AP$, $BP$, $CP$ meet the circumcircle again at $T$, $S$, $R$ (here $T \neq A$, $S \neq B$, $R \neq C$). Let $U$ be any point in the interior of $PT$. A line through $U$ parallel to $AB$ meets $CR$ at $W$, and the line through $U$ parallel to $AC$ meets $BS$ again at $V$. Finally, the line through $B$ parallel to $CP$ and the line through $C$ parallel to $BP$ intersect at point $Q$. Given that $RS$ and $VW$ are parallel, prove that $\angle CAP = \angle BAQ$.

  \item (\href{https://aops.com/community/c6h1226672}{EGMO 2016/2}) Let $ABCD$ be a cyclic quadrilateral, and let diagonals $AC$ and $BD$ intersect at $X$. Let $C_1$, $D_1$ and $M$ be the midpoints of segments $CX$, $DX$ and $CD$, respectively. Lines $AD_1$ and $BC_1$ intersect at $Y$, and line $MY$ intersects diagonals $AC$ and $BD$ at different points $E$ and $F$, respectively. Prove that line $XY$ is tangent to the circle through $E,F$ and $X$.

  \item (\href{https://aops.com/community/c6h486936}{ELMO 2012/5}) Let $ABC$ be an acute triangle with $AB<AC$, and let $D$ and $E$ be points on side $BC$ such that $BD=CE$ and $D$ lies between $B$ and $E$. Suppose there exists a point $P$ inside $ABC$ such that $PD\parallel AE$ and $\angle PAB=\angle EAC$. Prove that $\angle PBA=\angle PCA$.
\end{enumerate}

\subsubsection*{Equal segments}

If you have two segments of the same length but they are far, a general way to take advantage is to construct a parallelogram. Sometimes, this makes an isosceles triangle or isosceles trapezoid.

\begin{enumerate}
  \item (\cite{A}) Let $ABCDE$ be a convex pentagon with $AB = BC$ and $CD = DE$. If $\angle ABC = 2\angle CDE = 120\dg$ and $BD = 2$, find the area of $ABCDE$. 
  % metric conditions beget parallelograms

  \item Let $ABC$ be an acute triangle with orthocenter $H$. Let $G$ be the point such that the quadrilateral $ABGH$ is a parallelogram. Let $I$ be the point on the line $GH$ such that $AC$ bisects $HI$. Suppose that the line $AC$ intersects the circumcircle of triangle $CGI$ at $C$ and $J$. Prove that $IJ = AH$. % draw parallelograms because of lengths

  \item (\href{https://aops.com/community/c6h1670580}{IMO 2018/1}) Let $\Gamma$ be the circumcircle of acute triangle $ABC$. Points $D$ and $E$ are on segments $AB$ and $AC$ respectively such that $AD = AE$. The perpendicular bisectors of $BD$ and $CE$ intersect minor arcs $AB$ and $AC$ of $\Gamma$ at points $F$ and $G$ respectively. Prove that lines $DE$ and $FG$ are either parallel or they are the same line.\footnote{For me, this feels conceptually closer to Problem 5 in section Grab Bag. Compare the solution posted on \url{http://aops.com/community/c6h1670580p10626602}.}
  % equal lengths beget parallelograms? again, not really. for me, this was conceptually ``arcs of the same length in a circle''

  \item (\href{https://aops.com/community/c6h1438017}{Iran TST3 2017/6}) Let $O$ and $H$ be the circumcenter and orthocenter of triangle $ABC$. Let $P$ be the reflection of $A$ with respect to line $OH$, and suppose that $P$ is not on the same side of line $BC$ as $A$. Points $E$ and $F$ lie on $AB$ and $AC$ respectively such that $BE = PC$ and $CF = PB$. Let $K$ be the intersection of lines $AP$ and $OH$. Prove that $\angle EKF = 90^{\circ}$. 
  % draw parallelograms because of lengths
  % Parallelogram BPCR, <ERF right, R-O-H, A-L-L'-K-P with a bunch more parallelograms
  % https://artofproblemsolving.com/community/c6h1438017p8160421
\end{enumerate}

\subsubsection*{Midpoints}

If you have a midpoint of a segment, try completing a parallelogram with that segment as a diagonal. Sometimes useful to convert to angle conditions, sometimes forms cyclic quadrilaterals and the like.

\begin{enumerate}
  \item (\cite{A}) Let $ABC$ be a triangle and let $M$ be the midpoint of $BC$. Squares $ABQP$ and $ACYX$ are constructed externally. Show that $PX = 2AM$.
  % midpoints and angles -> parallelograms

  \item (\href{https://aops.com/community/c6h53670}{USAMO 2003/4}) Let $ABC$ be a triangle. A circle passing through $A$ and $B$ intersects segments $AC$ and $BC$ at $D$ and $E$, respectively. Lines $AB$ and $DE$ intersect at $F$, while lines $BD$ and $CF$ intersect at $M$. Prove that $MF = MC$ if and only if $MB\cdot MD = MC^2$. 
  % midpoints

  \item (\href{https://aops.com/community/c139h555561}{NIMO 8/8}) The diagonals of convex quadrilateral $BSCT$ meet at the midpoint $M$ of $\overline{ST}$. Lines $BT$ and $SC$ meet at $A$, and $AB = 91$, $BC = 98$, $CA = 105$. Given that $\overline{AM} \perp \overline{BC}$, find the positive difference between the areas of $\triangle SMC$ and $\triangle BMT$. 
  % midpoints

  \item (Euler's quadrilateral theorem) In quadrilateral $ABCD$, let $M$ and $N$ be the midpoints of diagonals $AC$ and $BD$, respectively. Then $AB^2 + BC^2 + CD^2 + DA^2 = AC^2 + BD^2 + 4MN^2$.

  \item (\href{https://aops.com/community/c6h355791}{ISL 2009/G4}) Given a cyclic quadrilateral $ABCD$, let the diagonals $AC$ and $BD$ meet at $E$ and the lines $AD$ and $BC$ meet at $F$. The midpoints of $AB$ and $CD$ are $G$ and $H$, respectively. Show that $EF$ is tangent at $E$ to the circle through the points $E$, $G$ and $H$. 
  % midpoints and angles -> parallelograms? not really...
\end{enumerate}

\subsubsection*{Tangents}

These two problems have the same configuration and are conceptually similar in terms of the parallelogram to complete.

\begin{enumerate}
  \item (\href{https://aops.com/community/c6h1480682}{IMO 2017/4}) Let $R$ and $S$ be different points on a circle $\Omega$ such that $RS$ is not a diameter. Let $\ell$ be the tangent line to $\Omega$ at $R$. Point $T$ is such that $S$ is the midpoint of the line segment $RT$. Point $J$ is chosen on the shorter arc $RS$ of $\Omega$ so that the circumcircle $\Gamma$ of triangle $JST$ intersects $\ell$ at two distinct points. Let $A$ be the common point of $\Gamma$ and $\ell$ that is closer to $R$. Line $AJ$ meets $\Omega$ again at $K$. Prove that the line $KT$ is tangent to $\Gamma$. 
  % https://artofproblemsolving.com/community/c6h1480682p8639236

  \item (\href{https://aops.com/community/c6h1107238}{ELMO 2015/3}) Let $\omega$ be a circle and $C$ a point outside it; distinct points $A$ and $B$ are selected on $\omega$ so that $\overline{CA}$ and $\overline{CB}$ are tangent to $\omega$. Let $X$ be the reflection of $A$ across the point $B$, and denote by $\gamma$ the circumcircle of triangle $BXC$. Suppose $\gamma$ and $\omega$ meet at $D \neq B$ and line $CD$ intersects $\omega$ at $E \neq D$. Prove that line $EX$ is tangent to the circle $\gamma$.
  % https://artofproblemsolving.com/community/c6h1107238p5022031
\end{enumerate}

\subsection{Halving segments and angles}

If you have one segment twice another, like $AB = 2XY$, construct the midpoint of $XY$. If you have one angle twice another, like $\angle ABC = 2\angle XYZ$, construct the angle bisector of $\angle XYZ$. This gives you three equal segments or three equal angles to work with. Sometimes helpful.

\begin{enumerate}
  \item (IGO 2015/E2) Let $ABC$ be a triangle with $\angle A = 60^{\circ}$. The points $M$, $N$, and $K$ lie on $BC$, $AC$, and $AB$ respectively such that $BK = KM = MN = NC$. If $AN = 2AK$, determine the angles of triangle $ABC$.

  \item (\href{https://aops.com/community/c6h1710019}{IGO 2018/I2}) In convex quadrilateral $ABCD$, the diagonals $AC$ and $BD$ meet at $P$. Suppose $\angle DAC = 90^{\circ}$ and $2\angle ADB = \angle ACB$. If we have $\angle DBC + 2\angle ADC = 180^{\circ}$, prove that $2AP = BP$.

  \item (\cite{D}) In isosceles triangle $ABC$, $AB = AC$. Let $D$ be on side $BC$ such that $BD = 2DC$. Point $P$ lies on segment $AD$ such that $\angle ABP = \angle PAC$. Prove that $\angle BAC = 2\angle DPC$.

  \item (\href{https://aops.com/community/c6h61956}{USA TST 2001/5}) In triangle $ABC$, $\angle B = 2\angle C$. Let $P$ and $Q$ be points on the perpendicular bisector of segment $BC$ such that rays $AP$ and $AQ$ trisect $\angle A$. Prove that $PQ < AB$ if and only if $\angle B$ is obtuse.
\end{enumerate}

\subsection{Incenters and excenters}

There doesn't seem to be any similarities between these problems, other than the fact they can be solved by constructing incenters or excenters. Sometimes the center already exists as a point, and identifying it makes the problem easier.

\begin{enumerate}
  \item In triangle $ABC$, $\angle BAC = 120\dg$. The bisectors of the angles $\angle BAC, \angle ABC,$ and $\angle BCA$ intersect the opposite sides at the points $D, E$ and $F$, respectively. Prove that the circle with diameter $EF$ passes through $D$.

  \item (\href{https://aops.com/community/c123h515745}{USAMTS 2012/3/3}) In quadrilateral $ABCD$, $\angle DAB = \angle ABC = 110\dg$, $\angle BCD = 35\dg$, $\angle CDA = 105\dg$, and $AC$ bisects $\angle DAB$. Find $\angle ABD$.

  \item (\href{https://aops.com/community/c6h404325}{All-Russian 2011/10/4}) Triangle $ABC$ has perimeter $4$. Points $X$ and $Y$ lie on rays $AB$ and $AC$ respectively such that $AX = AY = 1$. Segments $BC$ and $XY$ intersect at point $M$. Prove that the perimeter of either triangle $ABM$ or triangle $ACM$ is $2$.\footnote{Fun fact: this is often wrongly cited as Russia 2010.}

  \item (\href{https://aops.com/community/c6h355790}{ISL 2009/G3}) Let $ABC$ be a triangle. The incircle of $ABC$ touches the sides $AB$ and $AC$ at the points $Z$ and $Y$, respectively. Let $G$ be the point where the lines $BY$ and $CZ$ meet, and let $R$ and $S$ be points such that the two quadrilaterals $BCYR$ and $BCSZ$ are parallelogram. Prove that $GR=GS$.

  \item (\href{https://aops.com/community/c6h54506}{USAMO 1999/6}) Let $ABCD$ be an isosceles trapezoid with $AB \parallel CD$. The inscribed circle $\omega$ of triangle $BCD$ meets $CD$ at $E$. Let $F$ be a point on the (internal) angle bisector of $\angle DAC$ such that $EF \perp CD$. Let the circumscribed circle of triangle $ACF$ meet line $CD$ at $C$ and $G$. Prove that the triangle $AFG$ is isosceles. 

  \item (\href{https://aops.com/community/c6h1671264}{ISL 2017/G1}) Let $ABCDE$ be a convex pentagon such that $AB=BC=CD$, $\angle{EAB}=\angle{BCD}$, and $\angle{EDC}=\angle{CBA}$. Prove that the perpendicular line from $E$ to $BC$ and the line segments $AC$ and $BD$ are concurrent. 

  \item (\href{https://aops.com/community/c7419h1138166}{Geolympiad Summer 2015/5}) Let $ABC$ be a triangle and $P$ be in its interior. Let $Q$ be the isogonal conjugate of $P$. Show that $BCPQ$ is cyclic if and only if $AP=AQ$. 

  \item (\href{https://aops.com/community/c6h14092}{IMO 2004/5}) In a convex quadrilateral $ABCD$, the diagonal $BD$ bisects neither the angle $ABC$ nor the angle $CDA$. The point $P$ lies inside $ABCD$ and satisfies $\angle PBC=\angle DBA$ and $\angle PDC=\angle BDA$. Prove that $ABCD$ is a cyclic quadrilateral if and only if $AP=CP$.
\end{enumerate}

\subsection{Supplementary angles make two similar triangles}

An oddly specific construction, which is kind of hard to tell when needed, involves a cyclic quadrilateral $ABCD$. We construct the point $P$ on the segment $AC$ such that $BP$ and $BD$ are isogonal. Then triangles $ABP$ and $DBC$ are similar, and triangles $PBC$ and $ABD$ are similar as well. Sometimes can be applied without cyclic quadrilaterals.

\begin{enumerate}
  \item (Ptolemy's theorem) In cyclic quadrilateral $ABCD$, $AC \cdot BD = AB \cdot CD + BC \cdot DA$.

  \item (Generalization, Philippines 2019) In triangle $ABC$, $D$ and $E$ are points on sides $AB$ and $AC$ respectively. Point $Y$ is in triangle $ABC$ such that $\angle DYB$ and $\angle EYC$ are supplementary. Let $X$ be a point inside the triangle such that $\angle XBC = \angle EBA$ and $\angle XCB = \angle DCA$. Prove that $\angle BAC$ and $\angle EXD$ are complementary.
    
  \item (\href{https://aops.com/community/c6h1703436}{IOM 2018/6}) The incircle of a triangle $ABC$ touches the sides $BC$ and $AC$ at points $D$ and $E$, respectively. Suppose $P$ is the point on the shorter arc $DE$ of the incircle such that $\angle APE = \angle DPB$. The segments $AP$ and $BP$ meet the segment $DE$ at points $K$ and $L$, respectively. Prove that $2KL = DE$.
\end{enumerate}

\subsection{Grab bag}

These problems all involve a construction as a key step in their solution. 

\begin{enumerate}
  \item (\href{https://aops.com/community/c6h77707}{Canada 2000/4}) Let $ABCD$ be a convex quadrilateral with $\angle CBD = 2 \angle ADB$, $\angle ABD = 2 \angle CDB$ and $AB = CB$. Prove that $AD = CD$.

  \item (\href{https://aops.com/community/c6h529042}{EGMO 2013/1}) The side $BC$ of triangle $ABC$ is extended beyond $C$ to $D$ so that $CD = BC$. The side $CA$ is extended beyond $A$ to $E$ so that $AE = 2CA$. Prove that if $AD = BE$ then triangle $ABC$ is right angled.
  
  \item (\href{https://aops.com/community/c6h228384}{Italy TST 2001/1}) The diagonals $ AC$ and $ BD$ of a convex quadrilateral $ ABCD$ intersect at point $ M$. The bisector of $ \angle ACD$ meets the ray $ BA$ at $ K$. Given that $ MA \cdot MC +MA \cdot CD = MB \cdot MD$, prove that $ \angle BKC = \angle CDB$.
  % multiplicative conditions beget POP
  % https://artofproblemsolving.com/community/c6h228384p1265388

  \item (\href{https://aops.com/community/c6h326960}{USA TST 2000/2}) Let $ ABCD$ be a cyclic quadrilateral and let $ E$ and $ F$ be the feet of perpendiculars from the intersection of diagonals $ AC$ and $ BD$ to $ AB$ and $ CD$, respectively. Prove that $ EF$ is perpendicular to the line through the midpoints of $ AD$ and $ BC$.

  \item On arc $BC$ of the circumcircle of triangle $ABC$, two points $X$ and $Y$ are chosen such that they both lie on the side of $BC$ that is opposite of $A$, and $\angle BAX = \angle CAY$. Let $M$ be the midpoint of chord $AX$. Show that $BM + CM \geq AY$.\footnote{Compare with \href{https://aops.com/community/c6h1670580}{IMO 2018/1}; in particular \href{http://artofproblemsolving.com/community/c6h1670580p10626602}{this solution} on AoPS.} % THIS ONE IS ACTUALLY CONSTRUCT AN ISOSCELES TRAPEZOID >:( 

  \item In triangle $ABC$, $\angle BCA = \angle CAB + 90\dg$. Point $D$ is on ray $BC$ such that $AC = AD$. Let point $E$ be such that $\angle EBC = \angle CAB$ and $2\angle EDC = \angle CAB$. Prove that $\angle CED = \angle ABC$.

  \item In triangle $ABC$, point $D$ lies on line $BC$ such that $C$ is between $B$ and $D$. Suppose there exists a unique point $X$ on line $AD$ such that $AX/BX = CX/DX$. Prove that triangle $ABC$ is isosceles.

  \item (\href{https://aops.com/community/c6h1437519}{Iran TST3 2017/2}) Let $P$ be a point in the interior of quadrilateral $ABCD$ such that: $\angle BPC = 2\angle BAC$, $\angle PCA = \angle PAD$, and $\angle PDA = \angle PAC$. Prove that $\angle PBD = |\angle BCA - \angle PCA|$.
% https://artofproblemsolving.com/community/c6h1437519p8153735

  \item (\href{https://aops.com/community/c6h1671293}{IMO 2018/6}) A convex quadrilateral $ABCD$ satisfies $AB\cdot CD = BC\cdot DA$. Point $X$ lies inside $ABCD$ so that $\angle{XAB} = \angle{XCD}$ and $\angle XBC = \angle XDA$. Prove that $\angle{BXA} + \angle{DXC} = 180\dg$.
  % construct isogonal conjugate of X
  % https://artofproblemsolving.com/community/c6h1671293p10632360
\end{enumerate}

\subsubsection*{Adventitious quadrangles}

Classical; don't take these too seriously. A general method for solving these problems exist, see \cite{B}.

\begin{enumerate}
  \item In isosceles triangle $ABC$, $AB = AC$ and $\angle BAC = 20\dg$. Points $D$ and $E$ are on $AC$ and $AB$ respectively such that $\angle CBD = 40\dg$ and $\angle BCE = 50\dg$. Determine $\angle CED$.

  \item In isosceles triangle $ABC$, $AB = AC$ and $\angle BAC = 20\dg$. Points $D$ and $E$ are on $AC$ and $AB$ respectively such that $\angle CBD = 50\dg$ and $\angle BCE = 60\dg$. Determine $\angle CED$.

  \item In isosceles triangle $ABC$, $AB = AC$ and $\angle BAC = 20\dg$. Points $D$ and $E$ are on $AC$ and $AB$ respectively such that $\angle CBD = 60\dg$ and $\angle BCE = 70\dg$. Determine $\angle CED$.

  % 2008 AMC 10B 24
  \item In convex quadrilateral $ABCD$, $AB = BC = CD$, $\angle ABC = 70\dg$, and $\angle BCD = 170\dg$. Determine $\angle DAB$.

  \item In convex quadrilateral $ABCD$, $\angle ABD = 12\dg$, $\angle ACD = 24\dg$, $\angle DBC = 36\dg$, and $\angle BCA = 48\dg$. Determine $\angle ADC$.

  \item In convex quadrilateral $ABCD$, $\angle ABD = 38\dg$, $\angle ACD = 48\dg$, $\angle DBC = 46\dg$, and $\angle BCA = 22\dg$. Determine $\angle ADC$.
\end{enumerate}

\newpage

\section{Hints}

\setlist[enumerate]{topsep=0pt,itemsep=-0.7ex,partopsep=1ex,parsep=1ex}

\subsubsection*{Same point on the same side}

\begin{enumerate}
  \item Construct on ray $AE$.
  \item Construct on ray $AB$.
  \item Construct on ray $BT$.
  \item Find a clever way to prove lengths of equal segments.
  \item Construct on rays $AC$ and $AB$.
  \item Construct on rays $AQ$ and $AB$.
\end{enumerate}

\subsubsection*{Same point on opposite sides}

\begin{enumerate}
  \item Construct on ray $AD$.
  \item Complete two isosceles triangles on $AB$.
  \item Construct on ray $BC$.
  \item Complete two isosceles triangles on $BC$.
  \item Complete two isosceles triangles on $BC$.
  \item Complete two isosceles triangles on $MN$.
  \item Construct on ray $AE$.
\end{enumerate}

\subsubsection*{Existing parallelogram or isogonality}

\begin{enumerate}
  \item Existing parallelogram $PBQC$, additional point $A$.
  \item Existing parallelogram $ABCD$, additional point $O$.
  \item Existing parallelogram $CEDG$, additional point $F$.
  \item Existing parallelogram $PBQC$, additional point $A$.
  \item Existing parallelogram $C_1MD_1X$, additional point $Y$.
  \item Complete $BPCQ$. Use as existing parallelogram, additional point $A$.
\end{enumerate}

\subsubsection*{Equal segments}

\begin{enumerate}
  \item Segment $AB$ is a side. Rearrange to make area easier to calculate.
  \item Segment $JI$ is a side; $H$ is a vertex.
  \item Segment $FD$ is a side; $A$ is a vertex. Similarly, segment $GE$ is a side; $A$ is a vertex.
  \item Segment $PC$ is a side; $B$ is a vertex.
\end{enumerate}

\subsubsection*{Midpoints}

\begin{enumerate}
  \item Segment $BC$ is a diagonal; $A$ is a vertex.
  \item Segment $CF$ is a diagonal; $D$ is a vertex.
  \item Segment $ST$ is a diagonal; $A$ is a vertex.
  \item Segment $AC$ is a diagonal; $B$ is a vertex. Also, segment $AC$ is a diagonal; $D$ is a vertex.
  \item Segment $CD$ is a diagonal; $F$ is a vertex. Similarly, segment $AB$ is a diagonal; $F$ is a vertex.
\end{enumerate}

\subsubsection*{Tangents}

\begin{enumerate}
  \item Complete parallelogram $RATP$.
  \item Complete parallelogram $AYXC$.
\end{enumerate}

\subsubsection*{Incenters and excenters}

\begin{enumerate}
  \item Construct $A$-excircle.
  \item Observe $C$ is $A$-excenter of $ABD$.
  \item Construct $A$-excircle.
  \item Construct $A$-excircle.
  \item Observe $F$ is $A$-excenter of $ACD$.
  \item Construct incenter.
  \item Construct incenter.
  \item Construct $B$- and $D$-excenters of $BPD$.
\end{enumerate}

\subsubsection*{Supplementary angles make two similar triangles}

\begin{enumerate}
  \item Construct $K$ on diagonal $AC$ such that $BK$ and $BD$ are isogonal.
  \item Construct $Z$ on side $BC$ such that $\angle XZB = \angle DYB$.
  \item Construct $X$ on diagonal $DE$ such that $PX$ and $PF$ are isogonal.
\end{enumerate}

\subsubsection*{Grab bag}

\begin{enumerate}
  \item Intersect opposite sides of $ABCD$.
  \item Observe $A$ is centroid of $EBD$.
  \item Draw circle with center $C$ passing through $D$. Intersect with line $AC$.
  \item Let $AC$ and $BD$ meet at $P$. Construct the midpoints of $AD$, $BC$, $DP$, and $AP$.
  \item Construct an isosceles trapezoid with base $AX$ and one vertex at $B$.
  \item Let $M$ be the midpoint of $CD$. Intersect $AM$ with $DE$ and $BE$.
  \item Intersect $AD$ with circumcircle of $ABC$.
  \item Let $T$ be the point such that $BPT$ and $APC$ are similar. Intersect $AC$ and $BT$.
  \item Construct the isogonal conjugate of $X$.
\end{enumerate}

No hints will be provided for \emph{Halving segments and angles} and \emph{Adventitious quadrangles}.

\begin{thebibliography}{99}

\bibitem{A} Evan Chen, \href{http://web.evanchen.cc/handouts/BMC_Parallelograms/BMC_Parallelograms.pdf}{All you have to do is construct a parallelogram!}

\bibitem{E} Evan Chen, Euclidean Geometry in Mathematical Olympiads.

\bibitem{B} Hiroshi Saito, \href{https://www.gensu.co.jp/saito/challenge/pdf/3circumcenter_d20180609.pdf}{Completion of finding proofs for generalized Langley's problems in
elementary geometry}.

\bibitem{C} Carlos Shine, Angle Chasing, MOP 2010.

\bibitem{D} Carlos Shine, Some Useful Constructions, MOP 2012.

\end{thebibliography}

Thanks Andrew Wu, Sharvil Kesarwani, Vincent Huang, Ankan Bhattacharya, Michael Diao, Sean Ty, and Albert Patupat for suggestions. If you have an addition or correction, please contact me at \mailto{cj@cjquines.com}.

\end{document}

% https://artofproblemsolving.com/community/c6h481933p2699668
% https://artofproblemsolving.com/community/c6h589676p9100795
% https://artofproblemsolving.com/community/c6h481898p2699453
% angle tricks? try to look at incenter excenter section and grab bag again
% those problems like "orthocenter of ABC lies on some line?" lot of times you construct some D on (ABC) and prove the reflections of D across sidelines of ABC are on the line
% https://artofproblemsolving.com/community/c6h545392p3153984
% https://artofproblemsolving.com/community/c6h490077p2747903
% IMO 2014/3
% Canada 2013/5
% https://artofproblemsolving.com/community/c6h452583p2544248
% midpoints: https://artofproblemsolving.com/community/c6h1388475p7730629
