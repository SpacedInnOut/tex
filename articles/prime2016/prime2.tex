\documentclass[10pt,paper=letter]{scrartcl}
\usepackage[alttitle]{cjquines}

\begin{document}

\title{VCSMS PRIME}
\subtitle{Session 2: Trigonometry}
\author{compiled by Carl Joshua Quines}
\date{September 23, 2016}

\maketitle

\subsubsection*{Circular functions}

\begin{enumerate}

\item (11QI8) Find the sum $\cos 1\dg + \cos 3\dg + \cos 5\dg + \cdots + \cos 177\dg + \cos 179\dg$.

\item (13QI15) Find the value of $\sin \theta$ if the terminal side of $\theta$ lies on $5y - 3x = 0$ and $\theta$ is in the first quadrant.

\item (11AI14) The line from the origin to the point $(1, \tan 75\dg)$ intersects the unit circle at $P$. Find the slope of the tangent line to the circle at $P$.

\item (11AI11) Find the sum of the coefficients of the polynomial $\cos(2 \cos^{-1}(1 - x^2))$.

\end{enumerate}

\subsubsection*{Identities}

\begin{enumerate}

\item (11QII5) Find the value of $\cos 15\dg$.

\item (14QII6) Evaluate $\log_2 \sin(\pi/8) + \log_2 \cos(15\pi/8)$.

\item (16NE9) If $\tan x + \tan y = 5$ and $\tan(x+y) = 10$, find $\cot^2 x + \cot^2 y$.

\item (15AI4) Find the numerical value of $(1-\cot 37\dg)(1-\cot 8\dg)$.

\item (16NA1) Find the value of $\cot (\cot^{-1} 2 + \cot^{-1} 3 + \cot^{-1} 4 + \cot^{-1} 5)$.

\item (16AI6) Evaluate $\displaystyle \prod_{\theta = 1}^{89} (\tan \theta\dg \cos 1\dg + \sin 1\dg).$

\item (13AI14) Given that $\tan \alpha + \cot \alpha = 4$, find $\sqrt{\sec^2 \alpha + \csc^2 \alpha - \frac{1}{2}\sec\alpha \csc \alpha}$.

% \item (AIME 2003/11) An angle $x$ is chosen at random from $0\dg < x < 90\dg$. Find the probability that the numbers $\sin^2 x, \cos^2 x$ and $\sin x \cos x$ are the lengths of the sides of a triangle.

\end{enumerate}

\subsubsection*{Equations}

\begin{enumerate}

\item (13QI11) If $2\sin(3x) = a\cos(3x + c)$, find all values of $ac$.

\item (13QI10) How many solutions has $\sin 2\theta - \cos 2\theta = \sqrt{6}/2$ in $\left( -\dfrac{\pi}{2}, \dfrac{\pi}{2}\right)$?

\item (10NA9) If $0 < \theta < \pi/2$ and $1 + \sin \theta = 2\cos \theta$, determine the numerical value of $\sin \theta$.

\item (13NE13) Find the solution set of the equation $\dfrac{\sec^2 x - 6\tan x + 7}{\sec^2 x - 5} = 2.$

\item (10ND4) Find the only value of $x$ in $(-\pi/2, 0)$ that satisfies $\dfrac{\sqrt{3}}{\sin x} + \dfrac{1}{\cos x} = 4.$

\item (16AI13) Find all real numbers $a$ and $b$ so that for all real numbers $x,$ $$2\cos^2\left(x+\frac{b}{2}\right)-2\sin\left(ax-\frac{\pi}{2}\right)\cos\left(ax-\frac{\pi}{2}\right) = 1.$$

\item (14AI12) Suppose $\alpha, \beta \in (0, \pi/2)$. If $\tan \beta = \dfrac{\cot\alpha - 1}{\cot\alpha + 1}$, find $\alpha + \beta$.

\item (14ND3) Find all $0 \leq \theta \leq 2\pi$ satisfying $\sqrt{\dfrac{1}{2} + \dfrac{1}{2}\sqrt{\dfrac{1}{2} + \dfrac{1}{2}\sqrt{\dfrac{1}{2} + \dfrac{1}{2}\cos 8\theta}}} = \cos \theta.$

\end{enumerate}

\subsubsection*{Triangle laws}

\begin{enumerate}

\item (16NE5) In right triangle $ABC, \angle ACB = 90\dg$ and $AC = BC = 1$. Point $D$ is on $AB$ such that $\angle DCB = 30\dg$. Find the area of $\triangle ADC$.

\item (13NE11) In $\triangle ABC, \angle A = 60\dg, \angle B = 45\dg,$ and $AC = \sqrt{2}$. Find the area of the triangle.

% \item (AIME 1985/9) In a circle, parallel chords of lengths $2, 3$ and $4$ determine central angles of $\alpha, \beta,$ and $\alpha + \beta$ radians, respectively, where $\alpha + \beta < \pi$. Find $\cos \alpha$.

% \item (AIME 2013/5) In equilateral triangle $ABC$, points $D$ and $E$ trisect $BC$. Find $\sin\angle DAE$.

\item (10QIII5) Let $M$ be the midpoint of side $BC$ of triangle $ABC$. Suppose that $AB = 4, AM = 1$. Determine the smallest possible measure of $\angle BAC$.

% \item (AIME 1988/7) In triangle $ABC, \tan \angle CAB = 22/7$, and the altitude from $A$ divides $BC$ into segments of length $3$ and $17$. What is the area of triangle $ABC$?

\item (13AI9) Consider an acute triangle with angles $\alpha, \beta, \gamma$ opposite the sides $a, b, c$ respectively. If $\sin \alpha = \dfrac{3}{5}$ and $\cos \beta = \dfrac{5}{13}$, evaluate $\dfrac{a^2+b^2-c^2}{ab}.$

\item (15AII3) Points $A, M, N$ and $B$ are collinear, in that order, and $AM = 4, MN = 2, NB = 3.$ If point $C$ is not collinear with these four points, and $AC = 6$, prove that $CN$ bisects $\angle BCM$.

% \item (AIME 1989/10) Let $a, b, c$ be three sides of a triangle, and $\alpha, \beta, \gamma$ be the angles opposite them. Given $a^2 + b^2 = 1989c^2,$ find $\dfrac{\cot\gamma}{\cot\alpha + \cot\beta}.$

\item (11AII2) Denote by $a, b, c$ the sides of a triangle opposite angles $\alpha, \beta, \gamma$, respectively. If $\alpha = 60\dg$, prove that $a^2 = \dfrac{a^3+b^3+c^3}{a+b+c}.$

\end{enumerate}

\end{document}