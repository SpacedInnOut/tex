\documentclass[10pt,paper=letter]{scrartcl}
\usepackage[alttitle]{cjquines}

\begin{document}

\title{VCSMS PRIME}
\subtitle{Session 2: Trigonometry}
\author{compiled by Carl Joshua Quines}
\date{September 23, 2016}

\maketitle

\subsubsection*{Circular functions}

\begin{enumerate}

\item As $\cos x = -\cos (180\dg - x)$, the sum is $0$.

\item Rearranging, $x/y = 5/3 = \tan \theta$. Thus $\sin \theta = 5/\sqrt{34}$.

\item The line is the terminal side of an angle $\theta$. Note that $\tan \theta = \tan 75\dg$, so the angle is $75\dg$. The tangent line to the unit circle makes an angle of $165\dg$ with the origin, so its slope is $\tan 165\dg = -2 + \sqrt{3}$.

\item We let $x = 1$ to get the sum of the coefficients as $\cos(2 \cos^{-1}(0)) = -1$.

\end{enumerate}

\subsubsection*{Identities}

\begin{enumerate}

\item The half-angle identity gives $\cos 15\dg = \dfrac{\sqrt{6}+\sqrt{2}}{4}$.

\item We wish to evaluate $\log_2 \sin(\pi/8)\cos(15\pi/8)$. By the product-to-sum identity, this is $\log_2 (1/2)(\sin(2\pi) + \sin(7\pi/4)) = -3/2$.

\item We use the fact that $\tan(x+y) = \dfrac{\tan x + \tan y}{1 - \tan x \tan y}$ to get $\tan x \tan y = \dfrac{1}{2}$. Then $\cot^2 x + \cot^2 y = \dfrac{(\tan x + \tan y)^2 - 2\tan x \tan y}{\tan^2 x \tan^2 y} = 96$.

\item Note that $\cot(37\dg + 8\dg) = \dfrac{\cot 37\dg \cot 8\dg - 1}{\cot 37\dg + \cot 8\dg} = 1$, so $\cot 37\dg \cot 8\dg - 1 = \cot 37\dg + \cot 8\dg$. This rearranges to $(1 - \cot 37\dg)(1 - \cot 8\dg) = 2$.

\item We see $\cot(\cot^{-1}2 + \cot^{-1}3) = \dfrac{2\cdot 3 - 1}{2 + 3} = 1$. Similarly, $\cot(\cot^{-1}4 + \cot^{-1}5) = 19/9$. Finally, $\cot(\cot^{-1}1 + \cot^{-1}19/9) = 5/14$.

\item Note that $\tan \theta\dg \cos 1\dg + \sin 1\dg = \dfrac{\sin\theta\dg \cos 1\dg + \sin 1\dg \cos \theta\dg}{\cos \theta\dg} = \dfrac{\sin(\theta\dg + 1\dg)}{\cos \theta\dg}$. The product telescopes using cofunctions and the result is $\dfrac{1}{\sin 1\dg} = \csc 1\dg$.

\item Interpret this with the unit circle: there is a right triangle with legs of length $\sec \alpha$ and $\csc \alpha$, and its hypotenuse is $\tan \alpha + \cot \alpha$. The area of the triangle is equal to half the product of its legs, or $\frac{1}{2}\sec \alpha \csc \alpha$. It is also equal to half the product of the hypotenuse and the altitude to the hypotenuse, or $\frac{1}{2}(\tan \alpha + \cot \alpha)$. The answer is $\sqrt{14}$.

\end{enumerate}

\subsubsection*{Equations}

\begin{enumerate}

\item (The equation holds for all $x$.) By phase shift, $2\sin 3x = 2\cos \del{3x - \dfrac\pi2 + 2k\pi} = -2\cos\del{3x + \dfrac\pi2 + 2k\pi}$ for some $k \in \ZZ$. The product $ac$ in both cases is $(4k-1)\pi$.

\item Square both sides to yield $1 - 2\sin2\theta\cos2\theta = 1 - \sin4\theta = 3/2$, giving $\sin 4\theta = -1/2$. Since $\theta \in \left(-\dfrac{\pi}{2}, \dfrac{\pi}{2}\right)$, it follows $4\theta \in (-2\pi, 2\pi)$. In this interval, $\sin 4\theta$ becomes $-1/2$ four times, so the equation has four solutions.

\item Square both sides and substitute $\cos^2\theta = 1 - \sin^2\theta$ to yield $5\sin^2\theta + 2\sin\theta - 3 = (5\sin\theta - 3)(\sin\theta + 1) = 0$. Either $\sin\theta = 3/5$ or $\sin\theta = -1$, but we can eliminate the latter as $0 < \theta < \pi/2$. Thus $\sin \theta = 3/5$.

\item Substituting $\sec^2x = \tan^2x + 1$ and simplifying gives the quadratic equation $\tan^2x + 6\tan x - 16 = (\tan x + 8)(\tan x - 2) = 0$, thus $x \in \{\tan^{-1}2\pm k\pi, \tan^{-1}(-8)\pm k\pi | k \in \ZZ\}$.

\item Transpose $\dfrac1{\cos x}$ and square both sides. Substitute $\sin^2x = 1 - \cos^2x$ and then $\cos x = u$ to get the equation $\dfrac3{1-u^2} = 16 + \dfrac1{u^2} - \dfrac8u$. Clear the denominators to get $16u^4 - 8u^3 - 12u^2 + 8u - 1 = 0$.

By inspection, $u = \dfrac12$ works; dividing through gives $8u^3 - 6u + 1 = 0$. This reminds one of the triple angle formula $\cos 3x = 4\cos^3x - 3\cos x$. We rewrite the equation as $4u^3 - 3u = -\dfrac12 = \cos3x$. Keeping in mind $x \in (-\pi/2, 0)$, we let $3x = -\dfrac{4\pi}3$ and get $x = -\dfrac{4\pi}9$.

\item Transpose the first term of the left hand side, use the double angle formulae, and then use cofunctions to get $\cos(2x + b) = \sin(2ax - \pi) = \cos(3\pi/2 - 2ax)$. We can see that there are two cases: when $a = 1$ and $b = \pi/2 + 2k\pi, k \in \ZZ$, or when $a = -1$ and $b = 3\pi/2 + 2k\pi, k \in \ZZ$.

\item Substitute $\cot \alpha = \dfrac{1}{\tan \alpha}$ and simplify to get $\tan \beta = \dfrac{1 - \tan\alpha}{1 + \tan\alpha}$. Cross-multiply and rearrange the terms to get $\tan \alpha + \tan \beta = 1 - \tan\alpha\tan\beta$, which is $\dfrac{\tan\alpha + \tan\beta}{1 - \tan\alpha\tan\beta} = \tan(\alpha + \beta) = 1$, so $\alpha + \beta = \pi/4$.

\item Note $\cos 8\theta = 2\cos^2 4\theta - 1$, so $\dfrac{1}{2} + \dfrac{1}{2}\cos 8\theta = \cos^2 4\theta$. Taking the positive root and repeating gives $\cos \theta$. Thus $\cos 4\theta, \cos 2\theta$ and $\cos \theta$ must all be at least $0$. This is when $\theta \in \sbr{0, \dfrac\pi8} \cup \sbr{\dfrac{15\pi}8, 2\pi}$.

\end{enumerate}

\subsubsection*{Triangle laws}

\begin{enumerate}

\item This is a $45\dg-45\dg-90\dg$ triangle, thus $\angle ACD = 60\dg$ and $\angle CDA = 75\dg$. By the sine law, $\dfrac{CD}{\sin 45\dg} = \dfrac{AC}{\sin 75\dg}$, so $CD = \sqrt{3} - 1$. The altitude of $ADC$ with respect to the base $AC$ has length $CD\sin 60\dg = \dfrac{1}{2}(3 - \sqrt{3})$, thus the area is $\dfrac{1}{4}(3 - \sqrt{3})$.

\item There is a solution with the sine law, but the synthetic solution involves letting $D$ be the foot of the altitude from $C$ to $AB$, making $ADC$ a $30\dg-60\dg-90\dg$ triangle and $BCD$ a $45\dg-45\dg-90\dg$ triangle. $AD$ has length $\dfrac{\sqrt{2}}{2}$ and $CD$ and $BD$ both have length $\dfrac{\sqrt{6}}{2}$. The area is then $\dfrac{3 + \sqrt{3}}{2}$.

\item Let $BM = MC = x$. By Apollonius', $AC = \sqrt{2x^2 - 14}$. We use the cosine law to get $\cos \angle BAC = \dfrac{4^2 - \del{\sqrt{2x^2 - 14}}^2 - \del{2x}^2}{2\cdot4\sqrt{2x^2-14}} = \dfrac{1 - x^2}{4\sqrt{2x^2-14}}$. We want to maximize this, and upon seeing the numerator being negative, we are inspired to take the negative and minimize using AM-GM. Then $\cos \angle BAC = -\dfrac1{4\sqrt2}\del{\dfrac{x^2 - 1}{\sqrt{x^2 - 7}}} = -\dfrac1{4\sqrt2}\del{\dfrac{x^2 - 7}{\sqrt{x^2 - 7}} + \dfrac6{\sqrt{x^2-7}}} = -\dfrac1{4\sqrt2}\del{\sqrt{x^2 - 7} + \dfrac6{\sqrt{x^2-7}}} \leq -\dfrac1{4\sqrt2}\cdot 2\sqrt6 = -\dfrac{\sqrt3}2$ by AM-GM. Thus $\angle BAC \geq 150\dg$.

\item By the cosine law, $\dfrac{a^2 + b^2 - c^2}{ab} = 2\cos \gamma$. Since $2\cos \gamma = 2\cos(\pi - \alpha - \beta) = -2\cos(\alpha + \beta)$, we can use the sum formula for cosine to get the answer as $\dfrac{32}{65}$.

\item There is a straightforward solution with the sine law, but we will proceed synthetically. Let $A'$ be the point on the line $AB$ that is not $N$ such that $A'A = 6$. Then $AA' = AC = AN = 6$, thus $A$ is the center of a circle with diameter $A'N$ containing point $C$, and $\angle A'CN = 90\dg$. Draw a line through $N$ parallel to $CA'$ and let it intersect lines $CM$ and $CB$ at $P$ and $Q$ respectively. Since $\triangle A'MC \sim \triangle PMN$ and $\triangle A'BC \sim \triangle NBQ$, we have $PN = \dfrac{MN}{MA'} \cdot CA'$ and $QN = \dfrac{BN}{BA'} \cdot CA'$, and substituting the given shows that $PN = QN$, which implies $\triangle CNP \cong \triangle CNQ$, which implies $\angle MCN = \angle NCB$.

\item By the cosine law, $a^2 = b^2 + c^2 - bc$. Factoring, $b^3 + c^3 = (b + c)(b^2 + c^2 - bc) = (b + c)a^2$. Add $a^3$ to both sides and rearrange to get the desired equality.

\end{enumerate}

\end{document}