\documentclass[10pt,paper=letter]{scrartcl}
\usepackage[alttitle]{cjquines}

\begin{document}

\title{VCSMS PRIME}
\subtitle{Session 5: Algebra 2}
\author{compiled by Carl Joshua Quines}
\date{October 5, 2016}

\maketitle

\subsubsection*{Equations}

\begin{enumerate}

\item $2/3$ of the work needs to be completed in $12$ days, so multiplying the amount of work by $2$ doubles the men, and multiplying the amount of days by $2/3$ multiplies the amount of men by $3/2$. There should be $180$ men to do the work, so there should be $180 - 60 = 120$ more workers.

\item Either $2 - x^2 = 1$ or $x^2 - 3\sqrt2x + 4 = 0$. The former when $x = \pm1$, the latter when $x = \sqrt2, 2\sqrt2$. However, $\sqrt2$ makes the base zero and thus undefined. The solutions are $x = -1, 1, 2\sqrt2$.

\item This is a linear equation, and by inspection $x = a + b + c$ satisfies, so it must be the only solution.

\item Observe that $\del{\sqrt{2014} + \sqrt{2013}}^{-\tan^2 x} = \del{\dfrac1{\sqrt{2014} + \sqrt{2013}}}^{\tan^2 x} = \del{\sqrt{2014} - \sqrt{2013}}^{\tan^2 x}$ after rationalizing the denominator. Equating the exponents gives $\tan^2 x = 3$, which is satisfied by $x = \dfrac\pi3, \dfrac{2\pi}3$.

\item Cross-multiplying and simplifying shows $x = \dfrac{2m-6}{m-5}$, which has its denominator is zero, or when $x = 2, 6$. This happens when $m = 5, 6$.

\item Let $u = 2015^x$ and note that $2015^{-x} = \dfrac1u$. Cross-multiplying and solving for $u$ yields $\sqrt{\dfrac{1-3k}{4-k}}$. The fraction has to be positive, which happens when $k < \dfrac13$ or $k > 4$.

\item Observe $x + 3 - 4\sqrt{x-1} = \del{\sqrt{x-1} - \sqrt4}^2$ and $x + 8 - 6\sqrt{x-1} = \del{\sqrt{x-1} - \sqrt9}^2$, so the LHS is $\abs{\sqrt{x-1} - 2} + \abs{\sqrt{x-1} - 3}$. There are three cases: $\sqrt{x-1} \geq 3$, $\sqrt{x-1} \leq 2$, and $2 \leq \sqrt{x-1} \leq 3$. Solving each case and taking the union gives any $x \in [5, 10]$ works.

\item Let $u = \sqrt{2} - 1$. Observe that $\dfrac1u = \dfrac1{\sqrt2 - 1} = \sqrt2 + 1$ upon rationalizing the denominator. Then $u^x + 8u^{-x} = 9$, or multiplying both sides by $u^x$, $u^{2x} + 8 = 9u^x$. This is quadratic in $u^x$, with solutions $1$ and $8$. Now $u^x = 1, 8$ yields $x = 0, \log_u 8$ or $\log_{\sqrt2 - 1} 8$.

\end{enumerate}

\subsubsection*{Systems of equations}

\begin{enumerate}

\item Taking the product of all equations and taking the cube root gives $wxyz = 30$. Dividing by the third and last equations gives $y = \dfrac23$ and $w = \dfrac52$. Thus $w + y = \dfrac{19}6$.

\item The first equation is $2xy - x + y - 6 = 0$, the second is $xy + x - y - 2 = 0$. Adding the two equations gives $xy = \dfrac83$, or $y = \dfrac8{3x}$. Substituting in either equation and solving for $x$ and $y$ gives $\del{-2, -\dfrac43}, \del{\dfrac43, 2}$.

\item Let $u = \sqrt{x+y}$. From $(x-y)(x+y) = 9$ we get $\sqrt{x-y} = \dfrac3u$, substituting in the first equation gives $u + \dfrac3u = 4$ or $u^2 - 4u + 3 = 0$ which has positive solution $u = 1$. Thus $x + y = 1$ and $x - y = 9$; adding and subtracting gives $(a, b) = (5, -4)$. Then $\dfrac{ab}{a+b} = -20$.

\item Subtracting the first from second equation gives $3w + 5x + 7y + 9z = 1$ and the second from third equation gives $5w + 7x + 9y + 11z = 5$. Subtracting these two from each other and dividing by $2$ gives $w + x + y + z = 2$.

\item Dividing both sides of the first equation by $4xy$ gives $\dfrac1x + \dfrac1y = \dfrac12$. Similarly, $\dfrac1y + \dfrac1z = \dfrac14$ and $\dfrac1z + \dfrac1x = \dfrac18$. Adding all the equations and dividing by $2$ gives $\dfrac1x + \dfrac1y + \dfrac1z = \dfrac7{16}$, subtracting the second equation gives $\dfrac1x = \dfrac3{16}$, so $x = \dfrac{16}3$.

\end{enumerate}

\subsubsection*{Complex numbers}

\begin{enumerate}

\item The polynomial factors as $(x+1)(x^2+1) = 0$ with roots $-1, i, -i$. Else, multiply both sides by $x-1$ to get $x^4 = 1$, the roots are the fourth roots of unity except for $1$, so $-1, i, -i$.

\item Since $i^4 = 1$, then $\dfrac1i \cdot \dfrac{i^3}{i^3} = i^3 = -i$. Similarly, the sum becomes $\del{1 - i - 1 + i + \cdots}$, with the pattern repeating. However, $1 - i - 1 + i = 0$, so every four terms cancel out, until $\dfrac1{i^{2012}}$, leaving $\del{1 - i - 1}^2 = -1$.

\item This is $1 + \dfrac2{z^4 - 1} = \dfrac i{\sqrt3}$, and thus $z^4 = -\dfrac12 - \dfrac {i\sqrt3}2 = \cis240\dg$. By de Moivre's, $z = \cis60\dg, \cis150\dg,$ $\cis240\dg, \cis330\dg$. Written out, $z = \dfrac12 + \dfrac{i\sqrt3}2, - \dfrac{\sqrt3}2 + \dfrac i2, - \dfrac12 - \dfrac{i\sqrt3}2, \dfrac{\sqrt3}2 - \dfrac i2$.

\item Since $z \neq 1$, we have $z^2 + z + 1 = 0$. Dividing both sides by $z$ shows $z + 1 + \dfrac1z = 0$, adding three to both sides shows $z + \dfrac1z + 4 = 3$.

\end{enumerate}

\subsubsection*{Polynomials}

\begin{enumerate}

\item We $P(-7) = a(-7)^7 + b(-7)^3 + c(-7) - 5$, or $7 + 5 = -a(7^7) - b(7^3) - c(7)$. Then $P(7) = a(7^7) + b(7^3) + c(7) - 5 = -(7 + 5) - 5 = -17$.

\item By Vieta's, making a mistake in the constant term means the sum of the roots is conserved, similarly making a mistake in the linear term means the product of the roots is conserved. The sum of the roots is $10$ and the product is $9$, meaning one such original equation can be $x^2 - 10x + 9 = 0$.

\item If the roots are $r, s$, the distance of the roots is $p = \abs{r - s}$. This is hard to work with, so we square the distance instead: $p^2 = \del{r - s}^2$. We can rewrite this in terms of the sum and product: $p^2 = \del{r + s}^2 - 4rs$, and then in terms of the coefficients using Vieta's: $p^2 = \del{\dfrac ba}^2 - \dfrac{4c}a$.

Turning $p \to 2p$ makes $p^2 \to 4p^2$, which is $4p^2 = 4\del{\dfrac ba}^2 - \dfrac{16c}a$. We also want to write this in terms of a shift in $c$, from the original. Suppose the roots are translated $k$ downward, then $c \to c - k$, making $4p^2 = \del{\dfrac ba}^2 - \dfrac{4\del{c-k}}a$. Equating and solving gives $k = \dfrac{3b^2}{4a} - 3c$.

\item The discriminant should be less than zero, which is $(4p)^2 - 4(4)(1 - q^2) < 0$, or $p^2 + q^2 < 1$. This is a circle of area $\pi$ over a square of area $4$, so the probability is $\dfrac\pi4$.

\item We can form $x^5$ through $(x)(x^2)^2$ or $(x)^3(x^2)$. The former term is $\binom{4}{1,1,2} (2)^1(-x)^1(x^2)^2 = -\dfrac{4!}{1!1!2!} 2x^5 = -24x^5$. The latter term is $\binom{4}{0,3,1} (2)^0(-x)^3(x^2)^1 = -4x^5$, and their sum gives the coefficient $-28$.

\item Consider $Q(x) = P(x) - 3$, which is a degree-four polynomial that attains its maximum value of $0$ at $x = 2$, $x = 3$. Thus $2$ and $3$ are both roots, and since they are maximums, they should have multiplicity 2. Thus $Q(x) = a(x-2)^2(x-3)^2$ for some constant $a$, since it is quartic. Then $P(x) = a(x-2)^2(x-3)^2 + 3$ and plugging in $x = 1$ gives $a = -\dfrac34$. Finally $P(5) = -24$.

\item Plugging in $x = 1$ gives $5^{2009} - 5^{2009} = 0$, the sum of the coefficients. Plugging in $x = -1$ gives $1^{2009} + 1^{2009} = 2$, which is the difference between the coefficients of terms with even exponents and coefficients of terms with odd exponents. So subtracting them will cancel out the terms with even exponents, and dividing by two gives the answer: $\dfrac{0 - 2}2 = -1$.

\item Treating this is $\del{x + (y + z)}^{2015} + \del{x - (y + z}^{2015}$ and expanding by the binomial theorem cancels out the terms with odd $(y + z)$ exponent, leaving $2x^{2015} + 2\binom{2015}2x^{2014}\del{y+z}^2 + \cdots + 2\binom{2015}{2014}x\del{y+z}^{2014}$. The first term has $1$ term, the second term has $3$ terms, etc.; none of these terms combine because they have different powers of $x$. All in all, there are $1 + 3 + \cdots + 2015 = 1016064$ terms.

\end{enumerate}

\subsubsection*{Polynomial factors}

\begin{enumerate}

\item Since $x^2 - x - 2 = (x - 2)(x + 1)$ and it divides $ax^4 + bx^2 + 1$, it must have $2$ and $-1$ as roots. Thus $16a + 4b + 1 = a + b + 1 = 0$. Solving yields $a = -\dfrac14$ and $b = -\dfrac34$.

\item The remaining factor must be $x^2 + cx + d$ for some $c, d$. Multiplying out gives $x^4 + (2 + c)x^3 + (2c + d + 5)x^2 + (5c + 2d)x + 5d$. Equating the cubic and linear coefficients shows $2 + c = 0$ and $5c + 2d = 0$, so $c = -2$ and $d = 5$. Then $a = 6$ and $b = 25$, so $a + b = 31$. 

Alternatively, note that since $x^4 + ax^2 + b$ can be rewritten as $\del{x^2 - h}^2 - k$, which is likely to be a difference of two squares. We guess it as $\del{(x^2 + 5) + 2x}\del{(x^2 + 5) - 2x}$, which gives the product $\del{x^2 + 5}^2 - 4x^2$, which fits the form. Substituting $1$ gives the sum of the coefficients, $6^2 - 4 = 32$, and subtracting the leading coefficient $1$ gives the answer $31$.

\item From $a^3 + b^3 + c^3 = 3abc$ if $a + b + c = 0$, we have $(r-s)^3 + (s-t)^3 + (t-r)^3 = 3(r-s)(s-t)(t-r)$. Alternatively, use the factor theorem by substituting $r = s$, to get $r - s$ is a factor, etc.

\item By the fundamental theorem of algebra, $x^{2015} + 18 = (x - r_1)(x - r_2) \cdots (x - r_{2015})$ for some complex roots $r_1, \ldots, r_{2015}$. Any combination of linear factors produces a factor of $x^{2015} + 18$.

Since each linear factor either appears or does not appear in the factor, there are $2^{2015}$ factors. However, we overcounted since one of them is where none of the linear factors appear, so there are $2^{2015} - 1$ factors.

\item Substituting $x = 5$ gives $p(4) = 0$. Then substituting $x = 4, 3, 2, 1$ in turn gives $p(3) = p(2) = p(1) = p(0) = 0$. Thus the polynomial $p(x) = x(x-1)(x-2)(x-3)(x-4)q(x)$ for some polynomial $q$. Substituting $p(x)$ in the original gives $q(x-1) = q(x)$, so $q(x)$ is constant. Substituting $x = 6$ gives the constant $\dfrac16$, so only $p(x) = \dfrac16x(x-1)(x-2)(x-3)(x-4)$ works.

\end{enumerate}

\subsubsection*{Remainder theorem}

\begin{enumerate}

\item By the remainder theorem, $P(r) = 2$. The division algorithm says $P(x) = (2x^2 + 7x - 4)(x-r)Q(x) + \del{-2x^2 - 3x + 4}$. Substituting $r$ gives $P(r) = 2 = -2r^2 - 3r + 4$, which has the solutions $r = -2, \dfrac12$.

\item By the remainder theorem, $f\del{-\dfrac32} = 4$ and $f\del{-\dfrac43} = 5$. By the division algorithm, $f(x) = (2x+3)(3x+4)Q(x) + R(x)$, and $R(x)$ is a linear polynomial. Substituing $x = -\dfrac32, -\dfrac43$ gives $R\del{-\dfrac32} = 4$ and $R\del{-\dfrac43} = 5$, which is the line $R(x) = 6x + 13$.

\item By a similar solution as above, the remainder is $R(x) = -x + 118$.

\end{enumerate}

\subsubsection*{Root-finding}

\begin{enumerate}

\item We solve the equation $2x^4 - 7x^3 + 2x^2 + 7x + 2 = 0$. Dividing both sides by $x^2$ and grouping terms gives $2\del{x^2 + \dfrac1{x^2}} - 7\del{x - \dfrac1x} + 2 = 0$. Letting $u = x - \dfrac1x$, observe $u^2 = x^2 - 2 + \dfrac1{x^2}$; substituting in gives $2\del{u^2 + 2} - 7u + 2 = 0$. This has roots $u = \dfrac32, 2$. 

The roots of $x - \dfrac1x = \dfrac32$ are $-\dfrac12$ and $2$, while the roots of $x - \dfrac1x = 2$ are $1 - \sqrt2$ and $1 + \sqrt2$. The smallest root is $-\dfrac12$ and the largest is $1 + \sqrt2$, and their difference is $\dfrac32 + \sqrt2$.

\item For the polynomial to have equal roots, the discriminant must be zero. Thus $\del{2(1+3m)}^2 - 4(7)(3 + 2m) = 0$, or $4(m-2)(9m + 10) = 0$, giving $m = -\dfrac{10}9, 2$. Substituting either shows that the equal root is $7$.

Alternatively, if the root is $r$, the polynomial must be $(x - r)^2 = x^2 - 2rx + r^2$. Equating coefficients gives $r = 1 + 3m$ and $r^2 = 7(3 + 2m)$. Then $\del{1 + 3m}^2 = 7(3 + 2m)$, again giving $m = -\dfrac{10}9, 2$, showing that $r = 7$.

\item The polynomial factors as $f(x-5) = -3(x-3)(x-12)$. Substituting $x = 3, 12$ shows that $f(-2) = f(7) = 0$, so its roots are $-2, 7$.

\item Suppose the roots are $a - d, a, a + d$. By Vieta's, the sum of the roots $3a = 6p$, so $a = 2p$, and our roots are $2p - d, 2p$ and $2p + d$. Using Vieta's on the linear and constant terms gives $-44 = p(4p^2 - d^2)$ and $5p = 12p^2 - d^2$. Solving for $d$ and equating, we get $4p^2 + \dfrac{44}p = 12p^2 - 5p$, or $8p^3 - 5p^2 - 44 = 0$. This factors as $(p-2)(8p^2 + 11p + 22) = 0$, and the only real root is $p = 2$.

\item By Vieta's, we know the sum of the roots is zero. Let the roots be $a, a, -2a$. The product of the roots is $-2a^3 = 128$, so $a = -4$. Then the sum of the pairwise products of the roots is $k = a^2 - 2a^2 - 2a^2 = -3a^2 = -48$.

\item By Vieta's, the product of the roots is $2$, so let the roots be $p, p, \dfrac2{p^2}$. Then the sum of pairwise products is $-3 = p^2 + \dfrac2p + \dfrac2p$, or $p^3 + 3p + 4 = 0$, which has the only real root $p = -1$. Thus the roots are $-1, -1, 2$ and $a = -(-1 -1 + 2) = 0$.

\end{enumerate}

\subsubsection*{Vieta's}

\begin{enumerate}

\item If the number is $x$, then $x - \dfrac1x = 2$, or $x^2 - 2x - 1 = 0$. By Vieta's, their product is $-1$.

\item WLOG the leading coefficient of the polynomial is one. Then $f(x) = x^{2016} - Sx^{2015} + \cdots$, so $f(2x - 3) = \del{2x - 3}^{2016} - S\del{2x - 3}^{2015} + \cdots$. Expanding and looking at the first two terms gives $(2x)^{2016} - 2016(2x)^{2015}(-3) - S(2x)^{2015} + \cdots$, or $2^{2016}x^{2016} - \del{2016\cdot2^{2015}\cdot3 + S\cdot2^{2015}}x^{2015}$. By Vieta's, the new sum of the roots is $\dfrac{S + 2016\cdot3}2 = \dfrac12S + 3024$.

Alternatively, $f(x)$ factors as $(x - r_1)\cdots(x-r_{2016})$ for its roots. Then $f(2x - 3) = (2x - 3 - r_1)\cdots(2x - 3 - r_{2016})$, which has roots $x = \dfrac{r_1 + 3}2, \dfrac{r_2 + 3}2, \ldots, \dfrac{r_{2016} + 3}2$. The new sum of the roots is then $\dfrac12S + 3024$.

\item Suppose the roots are $r$ and $s$. Then $\abs{r - s} = 75$, and squaring both sides gives $\del{r - s}^2 = 75^2$, or $r^2 - 2rs + s^2 = \del{r + s}^2 - 4rs = 75^2$. By Vieta's, $r + s = 51$ and so $rs = -\dfrac{75^2 - 51^2}4 = -756$. Then $r^2 + s^2 = \del{r + s}^2 - 2rs = 4113$.

\item Suppose the roots are $r$ and $s$. Then $r + s = -4$ and $rs = 8$ by Vieta's. The sum of the reciprocals of the roots is $\dfrac1r + \dfrac1s = \dfrac{r+s}{rs} = -\dfrac12$, and the product of the reciprocals is $\dfrac1r \cdot \dfrac1s = \dfrac18$. One such quadratic polynomial is thus $x^2 + \dfrac12x + \dfrac18$ by Vieta's, to make its coefficients integral we multiply by $8$ to get $8x^2 + 4x + 1 = 0$.

Alternatively, the transformation $x \to \dfrac1x$ makes the new polynomial have roots as reciprocals, giving $\dfrac1{x^2} + \dfrac4x + 8 = 0$. Multiplying by $x^2$ gives $8x^2 + 4x + 1 = 0$.

\item If the roots are $a, b, c, d$, then $\dfrac1a + \dfrac1b + \dfrac1c + \dfrac1d = \dfrac{abc + abd + acd + bcd}{abcd}$. By Vieta's, this is $\dfrac{-\frac24}{\frac{-6}4} = \dfrac13$.

Alternatively, the substitution $x \to \dfrac1x$ changes the roots to their reciprocals. The new polynomial is $\dfrac4{x^4} - \dfrac3{x^3} - \dfrac1{x^2} + \dfrac2x - 6 = 0$; multiplying both sides by $x^4$ gives $4 - 3x - x^2 + 2x^3 - 6x^4 = 0$. The sum of the roots is $-\dfrac2{-6} = \dfrac13$.

\item Both roots are real, so $b^2 \geq 4c$. In order to maximize $c$, we must have equality: let $b^2 = 4c$. We want to minimize $b + c = b + \dfrac{b^2}4 = \del{\dfrac b2 + 1}^2 -1 \geq 1$, which attains its minimum when $b = -2$.

\item Note that $x^3 - 4x + 1 = 0$ implies that each root satisfies $x^3 + 1 = 4x$. Substituting in the denominator and cancelling makes the sum $\dfrac{3abc}4$, which by Vieta's is $-\dfrac34$.

\end{enumerate}

\subsubsection*{Coordinate plane}

\begin{enumerate}

\item The first equation is two intersecting lines: $y = x$ and $y = -x$. The second equation is a circle of radius zero centered at $(a, 0)$, so it only consists of that point. Intersecting the line $y = 0$ with the first equation gives $x = 0$, so the only value of $a$ is $a = 0$.

\item The focus of $y = ax^2$ is $\del{0, \dfrac1{4a}}$, so the focus of $y = x^2 - 1$ is $\del{0, \dfrac14}$ shifted downward by one unit to $\del{0, -\dfrac34}$. Similarly, its vertex $\del{0, 0}$ is shifted downward to $\del{0, -1}$. We can rotate clockwise to $\del{-\dfrac14, -\dfrac34}$.

\item Since the parabola points upward, there must be either one or no intersections of the parabola with the line. Substituting $y = -12x + 5$ to $y = x^2 - 2px + p + 1$ gives $x^2 - (2p - 12)x + (p - 4) = 0$, and since it has at most one real root, its discriminant must be nonpositive. Thus $(2p + 12)^2 - 4(p - 4) \leq 0$, or $(p - 5)(p - 8) \leq 0$, which has solutions $p \in \sbr{5, 8}$.

\item Since the two circles are congruent, the two points of intersection must pass through the perpendicular bisector of their centers, $(0, 0)$ and $(16, 16)$. This perpendicular bisector is $x + y = 16$. Since $(a, b)$ and $(c, d)$ lie on this line, $a + b = 16$ and $c + d = 16$, so $a + b + c + d = 32$. 

\item The locus of points with $PA + PB = 10$ is an ellipse, the locus of point with $\abs{PC - PD} = 6$ is a hyperbola. The top-most point of the ellipse is $(0, 4)$ and the bottom-most point of the hyperbola is $(0, 3)$, so the upper arm of the hyperbola intersects it in two points. By symmmetry, so does the lower arm. Then there are $4$ points satisfying both.

\item The line passes through $P(0, 5)$ and is tangent to the circle with center $O(0, 0)$ and radius $3$. The point of tangency is $Q$, and since $\angle PQO = 90\dg$ due to tangency, $PQO$ is a $3-4-5$ right triangle.

We want to find the slope of $PQ$. Drop the perpendicular from $Q$ to $PO$ at point $R$, then the slope is $\dfrac{PR}{RQ}$, since it is rise over run. By similarity, $\dfrac{PR}{RQ} = \dfrac{PQ}{QO} = \dfrac43$.

\item Divide the plane in quadrants. We only need to consider $x > -15$, so this produces four equations: $x + y = \dfrac14\del{x + 15}$ in quadrant I, $-x + y = \dfrac14\del{x + 15}$ in quadrant II, $-x -y = \dfrac14\del{x+15}$ in quadrant III, and $x - y = \dfrac14\del{x+15}$ in quadrant IV. 

This produces a kite with vertices at the intercepts: $(5, 0)$, $\del{0, \dfrac{15}4}$, $(-3, 0)$ and $\del{0, -\dfrac{15}4}$. Thus its area is $30$.

\item Suppose $y = m(x-1)$. Substituting to the curve $4x^2 - y^2 - 8x = 12$ gives $(4 - m^2)x^2 + (2m^2 - 8)x - (m^2 + 12) = 0$. Since this should not have a solution, its discriminant should not be positive. Its discriminant is $(2m^2 - 8)^2 - 4(4 - m^2)(m^2 + 12) = 16(4 - m^2)$, which is positive when $m < 2$. The least positive $m$ is $2$, making the line $y = 2(x-1)$. 

\item Let $A(-2, 1), B(2, 5)$ and $C(5, 2)$. After either finding the side lengths or looking at the slopes, we notice that $\angle ABC = 90\dg$. The incenter must lie on the angle bisector of $\angle ABC$, which is a vertical line, so the incenter must be $(2, h)$ for some $h$. Since it is a right triangle, we can find its inradius using the formula $\dfrac{a + b - c}2 = \sqrt2$. 

This means the equation of the circle is $(x - 2)^2 + (y - h)^2 = 2$ for some $h$. Intersecting it with line $BC$ should only produce one intersection, since it is tangent. Line $BC$ is $y = 7-x$, substituting gives $2x^2 + (2h - 18)x + (h^2 - 14h + 51) = 0$, with discriminant $-4(h-3)(h-7)$. There should only be one solution, so the discriminant should be zero: we reject $h = 7$ because it is above the triangle. Thus the circle is $(x - 2)^2 + (y - 3)^2 = 2$.

\item Reflecting the point about the x-axis gives $P(-3, -7)$; the problem is now to find the shortest path from that point to the circle with center $O(5, 8)$. The segment $PO$ intersects the circle at $Q$, the shortest path has to be $PQ$. Then since $QO$ is a radius and has length $5$, we have $PQ = PO - QO = \sqrt{(-3 - 5)^2 + (-7 - 8)^2} - 5 = 12$.

\end{enumerate}

\end{document}