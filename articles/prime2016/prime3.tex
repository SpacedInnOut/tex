\documentclass[10pt,paper=letter]{scrartcl}
\usepackage[alttitle]{cjquines}

\begin{document}

\title{VCSMS PRIME}
\subtitle{Session 3: Number theory}
\author{compiled by Carl Joshua Quines}
\date{September 28, 2016}

\maketitle

\subsubsection*{Ad hoc}

\begin{enumerate}

\item (14QII8) How many trailing zeroes does $126!$ have when written in decimal notation?

\item (14AI7) What is the largest positive integer $k$ such that $27!$ is divisible by $2^k$?

\item (14NE2) What is the smallest number of integers that need to be selected from $\{1,2,\ldots,50\}$ to guarantee that two of the selected numbers are relatively prime?

\item (16QII4) How many positive integers $n$ make the expression $7^n + 7^3 + 2 \cdot 7^2$ a perfect square?

\item (11ND2) Find all positive integers $n$ make the expression $2^8 + 2^{11} + 2^n$ a perfect square.

\item (14NA7) What is the largest positive integer $abcdef$ that can be formed from the digits $1,2,3,4,5,6$, each used exactly once, if $abcdef$ is divisible by $6$, $abcde$ is divisible by $5$, $abcd$ is divisible by $4$, $abc$ is divisible by $3$, and $ab$ is divisible by $2$?

\item (14NA8) Let $N$ be the smallest integer such that the quotient $\dfrac{10!}{N}$ is odd. If $x$ and $y$ are nonnegative numbers such that $2x+y=N$, what is the maximum value of $x^2y^2$?

\item (16AI14) Let $P$ be the product of all prime numbers less than $90$. Find the largest integer $N$ so that for each $n \in \{2, 3, 4, \ldots, N\}$, the number $P+n$ has a prime factor less than $90$.

\end{enumerate}

\subsubsection*{Factors}

\begin{enumerate}

\item (16QI7) What is the fifth largest divisor of the number $2,015,000,000$?

\item (11NA9) What is the sum of all the even positive divisors of $1152$?

\item (15AI1) What is the fourth smallest positive integer having exactly $4$ positive integer divisors, including $1$ and itself?

\item (13NE4) If $25 \cdot 9^{2x} = 15^y$ is an equality of integers, what is the value of $x$?

\item (14QI14) How many factors of $7^{9999}$ are greater than $1,000,000$?

\item (13QI9) Determine the number of factors of $5^x + 2 \cdot 5^{x+1}$.

\item (14AI10) Let $p$ and $q$ be positive integers such that $pq = 2^3 \cdot 5^5 \cdot 7^2 \cdot 11$ and $\dfrac{p}{q} = 2 \cdot 5 \cdot 7^2 \cdot 11$. Find the number of positive integer divisors of $p$.

\item (11QIII3) Let $n = 2^{31}3^{19}$. How many positive divisors of $n^2$ are less than $n$ but do not divide $n$?

\item (16AI7) Find the sum of all the prime factors of $27,000,001$.

\item (9AI16) Give the prime factorization of $3^{20} + 3^{19} - 12$.

\item (16NA10) Let $m$ be the product of all positive integral divisors of $360,000$. Suppose the prime factors of $m$ are $p_1, p_2, \ldots, p_k$ for some positive integer $k$, and $m = p_1^{e_1}p_2^{e_2}\cdot \cdots \cdot p_k^{e_k},$ for some positive integers $e_1, e_2, \ldots, e_k$. Find $e_1 + e_2 + \cdots + e_k$.

\item (12N2) Let $f$ be a polynomial function with integer coefficients and $p$ be a prime number. Suppose that there are at least four distinct integers satisfying $f(x) = p$. Show that $f$ does not have integer zeros.

\end{enumerate}

\subsubsection*{Divisibility}

\begin{enumerate}

\item (9QII7) How many values of $n$ for which $n$ and $\dfrac{n+3}{n-1}$ are both integers?

\item (15AI11) Find all integer values of $n$ that will make $\dfrac{6n^3-n^2+2n+32}{3n+1}$ an integer.

\item (15AII2) What is the greatest common factor of all integers of the form $p^4 - 1$, where $p$ is a prime number greater than $5$?

\item (14AII2) Let $a, b,$ and $c$ be positive integers such that $\dfrac{a\sqrt{2013} + b}{b\sqrt{2013} + c}$ is a rational number. Show that $\dfrac{a^2+b^2+c^2}{a+b+c}$ and $\dfrac{a^3 - 2b^3 + c^3}{a+b+c}$ are both integers.

\item (11ND3) How many positive integer pairs $(x, y)$ are solutions to the equation $\dfrac{xy}{x+y}=1000$?

\item (16N2) Prove that the arithmetic sequence $5, 11, 17, 23, 29, \ldots$ contains infinitely many primes.

\end{enumerate}

\subsubsection*{Diophantine equations}

\begin{enumerate}

\item (9QI5) How many ordered pairs $(x, y)$ of positive integers satisfy $2x + 5y = 100$?

\item (16NE12) Find all values of integers $x$ and $y$ satisfying $2^{3x} + 5^{3y} = 189$.

\item (13NE3) Find the values of $a$ such that the system $x+2y = a+6, 2x - y = 25-2a$ has a positive integer pair solution $(x, y)$.

\item (14QII3) If $2xy + y = 43 + 2x$ for positive integers $x, y,$ find the largest value of $x + y$.

\item (14NE4) Find positive integers $a, b, c$ such that $a + b + ab = 15, b + c + bc = 99$ and $c + a + ca = 399$.

\item (11NA5) Find all nonnegative integer solutions of the system $5x+7y+5z=37, 6x-y-10z=3$.

\item (13NE8) If $7x + 4y = 5$ and $x$ and $y$ are integers, find the value of $\floor{y/x}$.

\item (14NE9) How many triples $(x, y, z)$ of positive integers satisfy the equation $x^{y^z} y^{z^x} z^{x^y} = 3xyz$?

\item (13N1) Determine, with proof, the least positive integer $n$ for which there exists $n$ distinct positive integers $x_1, x_2, x_3, \ldots, x_n$ such that $$\left(1- \frac{1}{x_1}\right) \left(1-\frac{1}{x_2}\right) \left(1 - \frac{1}{x_3} \right) \cdots \left(1 - \frac{1}{x_n}\right) = \frac{15}{2013}.$$

\end{enumerate}

\subsubsection*{Modulo}

\begin{enumerate}

\item (11AI4) Find the last $2$ nonzero digits of $16!$.

\item (16QI12) When $n+5$ is divided by $4$, the remainder is $3$. When $n+4$ is divided by $5$, the remainder is $5$. What is the remainder when $n+6$ is divided by $20$?

\item (16NE7) Let $n$ be a positive integer greater than $1$. If $2n$ is divided by $3$, the remainder $2$. If $3n$ is divided by $4$, the remainder is $3$. If $4n$ is divided by $5$, the remainder is $4$. If $5n$ is divided by $6$, the remainder is $5$. What is the least possible value of $n$?

\item (14NA2) Find the remainder when $3!^{5!^{7!^{\ldots^{2013!}}}}$ is divided by $11$.

\item (15AI17) What is the remainder when $16^{15} - 8^{15} - 4^{15} - 2^{15} - 1^{15}$ is divided by $96$?

\item (16AII1) The $6$-digit number $739ABC$ is divisible by $7, 8,$ and $9$. What values can $A, B,$ and $C$ take?

\item (13N4) Let $a, p,$ and $q$ be positive integers with $p \leq q$. Prove that if one of the numbers $a^p$ and $a^q$ is divisible by $p$, then the other number must also be divisible by $p$.

\end{enumerate}

\end{document}