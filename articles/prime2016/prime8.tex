\documentclass[10pt,paper=letter]{scrartcl}
\usepackage[alttitle]{cjquines}

\begin{document}

\title{VCSMS PRIME}
\subtitle{Session 8: Algebra 3}
\author{compiled by Carl Joshua Quines}
\date{October 14, 2016}

\maketitle

\subsubsection*{Manipulation}

\begin{enumerate}

\item (13NE10) If $\dfrac{x}{y} + \dfrac{y}{x} = 4$ and $xy = 3$, find the value of $xy(x+y)^2 - 2x^2y^2$.

\item (14NE1) If $x = \sqrt{2013-yz}, y = \sqrt{2014-zx}$ and $z = \sqrt{2015-xy}$, find $(x+y)^2 + (y+z)^2 + (z+x)^2$.

\item (14QIII4) If $m^3 - 12mn^2 = 40$ and $4n^3 - 3m^2n = 10,$ find $m^2 + 4n^2$.

\item (13AII1) If $x + y + xy = 1$, where $x, y$ are nonzero real numbers, find the value of $xy + \dfrac{1}{xy} - \dfrac{y}{x} - \dfrac{x}{y}$.

\item (14QIII3) If $\dfrac{a}{a^2 + 1} = \dfrac{1}{3}$, determine $\dfrac{a^3}{a^6 + a^5 + a^4 + a^3 + a^2 + a + 1}.$

\item (14QII9) It is known that $1 + \dfrac{1}{4} + \dfrac{1}{9} + \dfrac{1}{16} + \cdots = \dfrac{\pi^2}{6}$. Find the sum $1 + \dfrac{1}{9} + \dfrac{1}{25} + \dfrac{1}{49} + \cdots.$

\item (14AI18) Let $x$ be a real number so that $x + \dfrac{1}{x} = 3$. Find the last two digits of $x^{2^{2013}} + \dfrac{1}{x^{2^{2013}}}.$

\item (13ND5) Let $x = cy + bz, y = az + cx, z = bx + ay$. Find $\dfrac{(x-y)(y-z)(z-x)}{xyz}$ in terms of $a, b,$ and $c$.

\end{enumerate}

\subsubsection*{Surds}

\begin{enumerate}

\item (15AI6) Rationalize the denominator of $\dfrac{6}{\sqrt[3]{4} + \sqrt[3]{16} + \sqrt[3]{64}}$ and simplify.

\item (13QII10) If $36 - 4\sqrt{2} - 6\sqrt{3} + 12\sqrt{6} = (a\sqrt{2} + b\sqrt{3} + c)^2$, find the value of $a^2 + b^2 + c^2$.

\item (14ND2) Solve for $x$: $\sqrt{x+\sqrt{3x+6}} + \sqrt{x-\sqrt{3x+6}} = 6$.

\item (13NA6) If $\sqrt[3]{a+\sqrt{b}} = 12 + \sqrt{5},$ find the value of $\sqrt[3]{a-\sqrt{b}}$.

\item (11NA8) Let $a = \dfrac{\sqrt{5} + \sqrt{3}}{\sqrt{5} - \sqrt{3}}$ and $b = \dfrac{\sqrt{5} - \sqrt{3}}{\sqrt{5} + \sqrt{3}}$. Find the value of $a^4 + b^4 + (a + b)^4$.

\item (11AII3) Show that $\sqrt[n]{2} - 1 \leq \sqrt{\dfrac{2}{n(n-1)}}$ for all positive integers $n \geq 2$.

\end{enumerate}

\subsubsection*{Sequences}

\begin{enumerate}

\item (14AI4) The sequence $2, 3, 5, 6, 7, 8, 10, 11, \ldots$ is an enumeration of the positive integers that are not perfect squares. What is the $150$th term of this sequence?

\item (14QIII1) Let $\{ a_n\}$ be a sequence such that the average of the first and second terms is $1$, the average of the second and third terms is $2$, the average of the third and fourth terms is $3$, and so on. Find the average of the first and hundredth terms.

\item (11QII3) If $b_1 = \dfrac{1}{3}$ and $b_{n+1} = \dfrac{1-b_n}{1+b_n}$ for $n \geq 2$, find $b_{2010} - b_{2009}$.

\item (13AI8) Let $3x, 4y, 5z$ form a geometric sequence while $\dfrac{1}{x}, \dfrac{1}{y}, \dfrac{1}{z}$ form an arithmetic sequence. Find the value of $\dfrac{x}{z} + \dfrac{z}{x}$.

\item (16AII2) The numbers from $1$ to $36$ are written in a counterclockwise spiral as follows:

\begin{center}
\begin{tabular}{ccccc}
 \boxed{13} & 12 & 11 & 10 & 25 \\
14 & \boxed{3} & 2 & 9 & 24 \\
15 & 4 & \boxed{1} & 8 & 23 \\
16 & 5 & 6 & \boxed{7} & 22 \\
17 & 18 & 19 & 20 & \boxed{21} \\
\end{tabular}
\end{center}

In the figure above, all the terms on the diagonal beginning from the upper left corner have been enclosed in a box, and these entries sum up to $45$.

Suppose this spiral is continued all the way until $2015$, leaving an incomplete square. Find the sum of all the terms on the diagonal beginning from the upper left corner of the resulting (incomplete) square.

\item (14NE10) In the sequence $\{a_n\}, a_1 = 1, a_{n+1} = \dfrac{a_n}{1+ca_n}$ for some constant $c$. If $a_{12} = \dfrac{1}{2014}$, find $c$.

\item (9N1) In the sequence $\{a_n\}, n(n+1)a_{n+1} + (n-2)a_{n-1} = n(n-1)a_n$ for every positive integer $n$, where $a_0 = a_1 = 1$. Calculate the sum $\dfrac{a_0}{a_1} + \dfrac{a_1}{a_2} + \cdots + \dfrac{a_{2008}}{a_{2009}}$.

\end{enumerate}

\subsubsection*{Series}

\begin{enumerate}

\item (10QIII4) A sequence of consecutive positive integers beginning with $1$ is written on the blackboard. A student came along and erased one number. The average of the remaining numbers is $35 \frac{7}{17}$. What number was erased?

\item (14NA3) Let $n \leq m$ be positive integers such that the first $n$ numbers in $\{1,2,3,\ldots,m\}$ and the last $m-n$ numbers in the same sequence have the same sum $3570$. Find $m$.

\item (14QII7) If the sum of the infinite geometric series $\dfrac{a}{b} + \dfrac{a}{b^2} + \dfrac{a}{b^3} + \cdots$ is $4$, then what is the sum of $\dfrac{a}{a+b} + \dfrac{a}{(a+b)^2} + \dfrac{a}{(a+b)^3} + \cdots$?

\item (13NA4) Find the value of the infinite sum $1 + 1 + 3\left(\dfrac{1}{2}\right)^2 + 4\left(\dfrac{1}{2}\right)^3 + 5\left(\dfrac{1}{2}\right)^4 + \cdots$

\item (11NE8) Evaluate $\dfrac{1}{3} + \dfrac{1}{15} + \dfrac{1}{35} + \dfrac{1}{63} + \dfrac{1}{99} + \dfrac{1}{143} + \dfrac{1}{195}$.

\item (16AI10) Find the largest number $N$ so that $\displaystyle \sum_{n=5}^N \frac{1}{n(n-2)} < \frac{1}{4}$.

\item (16QII10) Find the sum of $\displaystyle \sum_{i=1}^{2015} \floor{\frac{\sqrt{i}}{10}}$.

\item (13AI7) Define $f(x) = \dfrac{a^x}{a^x + \sqrt{a}}$ for any $a > 0$. Evaluate $\displaystyle \sum_{i = 1}^{2012} f\left( \frac{i}{2013} \right)$.

\item (11NA6) Find the sum $\displaystyle \sum_{k=1}^{19} k \binom{19}{k}$.

\end{enumerate}

\subsubsection*{Inequalities}

\begin{enumerate}

\item (13AI18) Find all $k$ so that the inequality $k(x^2+6x-k)(x^2+x-12) > 0$ has solution set $(-4, 3)$.

\item (16NE14) Find the smallest $k$ such that for all real $x, y, z, (x^2+y^2+z^2)^2 \leq k(x^4 +y^4 + z^4)$.

\item (14NE14) Find the greatest $k$ such that for any positive real $a_1, a_2, a_3, a_4, a_5,$ with sum $S$, we have: $(S-a_1)(S-a_2)(S-a_3)(S-a_4)(S-a_5) \geq k(a_1a_2a_3a_4a_5).$

\item (12N3) If $ab > 0$ and $0 < x < \dfrac{\pi}{2}$, prove that $$\left(1 + \frac{a^2}{\sin x}\right) + \left(1 + \frac{b^2}{\cos x}\right) \geq \frac{(1+\sqrt{2}ab)^2\sin 2x}{2}.$$

\item (9N4) Let $k$ be a positive integer such that $\dfrac{1}{k+a} + \dfrac{1}{k+b} + \dfrac{1}{k+c} \leq 1$ for any positive real numbers $a, b, c$ with $abc = 1$. Find the minimum value of $k$.

\end{enumerate}

\subsubsection*{Single-variable extrema}

\begin{enumerate}

\item (13QIII2) Find the maximum of $y = (7 - x)^4(2+x)^5$ when $x$ lies strictly between $-2$ and $7$.

\item (14NE7) Find the maximum of $4x - x^4 - 1$.

\item (14NA10) Find the maximum of $\sqrt{(x-4)^2 + (x^3-2)^2} - \sqrt{(x-2)^2 + (x^3+4)^2}$.

\item (16NA2) Find the minimum of $x^2 + 4y^2 - 2x$, where $x, y$ are reals that satisfy $2x + 8y = 3$.

\item (13ND4) Find the minimum of $a^6 + a^4 - a^3 - a + 1$.

\end{enumerate}

\subsubsection*{Multi-variable extrema}

\begin{enumerate}

\item (10QI7) If $\abs{2x-3} \leq 5$ and $\abs{5-2y} \leq 3$ find the minimum of $x - y$.

\item (15AI16) Find the maximum of $\displaystyle \sum_{i=1}^{2014} (\sin \theta_i)(\cos \theta_{i+1})$, where $\theta_1 = \theta_{2015}$.

\item (8ND2) If $a$ and $b$ are positive real numbers, what is the minimum of $\sqrt{a+b}\left(\dfrac{1}{\sqrt{a}}+ \dfrac{1}{\sqrt{b}}\right)$?

\item (14QIII5) Find the minimum of $2a^8 + 2b^6 + a^4 - b^3 - 2a^2 - 2,$ where $a$ and $b$ are real numbers.

\item (10NA5) Let $x, y \in \RR^+$ such that $x + 2y = 8$. Determine the minimum value of $x + y + \dfrac{3}{x} + \dfrac{9}{2y}$.

\item (15AI19) Find the maximum of $(1-x)(2-y)(3-z)\left(x+\dfrac{y}{2}+\dfrac{z}{3}\right)$ where $x < 1, y < 2, z < 3,$ and $x + \dfrac{y}{2} + \dfrac{z}{3} > 0$.

\item (16AI18) Given $f(1-x) + (1-x)f(x) = 5$ for all real numbers $x$, find the maximum of $f(x)$.

\item (16ND2) Suppose $\dfrac{1}{2} \leq x \leq 2$ and $\dfrac{4}{3} \leq y \leq \dfrac{3}{2}$. Determine the minimum of $\dfrac{x^3y^3}{x^6 + 3x^4y^2 + 3x^3y^3 + 3x^2y^4 + y^6}.$

\item (13N5) Let $r$ and $s$ be positive real numbers that satisfy $(r+s-rs)(r+s+rs)=rs$. Find the minimum of $r+s-rs$ and $r+s+rs$.

\end{enumerate}

\end{document}