\documentclass[10pt,paper=letter]{scrartcl}
\usepackage[alttitle]{cjquines}

\begin{document}

\title{VCSMS PRIME}
\subtitle{Session 0: Introduction}
\author{Carl Joshua Quines}
\date{September 21, 2016}

\maketitle

\subsubsection*{Overview}

PRIME stands for Program for Improving Mathematical Excellence. A session will consist of roughly one to two hours of problem solving on the selected topic. Problem sets compiled from previous PMOs will be given out at the beginning of the session, and selected questions will be discussed. It is expected that most, if not all, of the problems in a given set should be solved within the week. If you do not know how to solve a problem on any problem set given out, do not hesitate to ask for help.

\subsubsection*{Schedule}

There will be ten sessions this year, followed by the PMO Qualifying Stage. Sessions start as soon as class ends, and will be held at the math lab. Take note that the GMATIC on October 19 is for the whole day, so no session will be held but the problem sets will still be given out.

\begin{enumerate}

\item Algebra 1, Wednesday, September 21

\item Trigonometry, Friday, September 23 

\item Number theory, Wednesday, September 28 

\item Combinatorics 1, Friday, September 30

\item Algebra 2, Wednesday, October 5

\item Combinatorics 2, Friday, October 7

\item Geometry 1, Wednesday, October 12

\item Algebra 3, Friday, October 14

\item Geometry 2, Wednesday, October 19

\item[*] GMATIC, Wednesday, October 19

\item Metasolving, Friday, October 21

\item[*] PMO Qualifying Stage, Saturday, October 22

\end{enumerate}

\subsubsection*{Notation}

The symbol $\RR^*$ is used for the non-zero real numbers, while $\RR^+$ is used for the positive real numbers. The natural numbers do not include zero.

The notation $\floor{x}$ is the floor function, or the greatest integer function, defined as the greatest integer less than or equal to $x$. We define $\{x\}$ as the fractional function, or $x - \floor{x}$, the fractional part of $x$. Finally, $\ceil{x}$ is the ceil function, or the least integer function, defined as the least integer greater than or equal to $x$.

Segments, rays and lines will not be marked with an overbar, and can be deduced from context.

\subsubsection*{Sources}

Problem sources are formatted in the following format:

\begin{itemize}

\item One or two digit year number

\item Stage: Q for Qualifying, A for Area, N for Nationals

\item Substage:

\begin{itemize}

\item For qualifying and area, I for test I, II for test II, III for test III

\item For national written, the substage is omitted.

\item For national orals, E for fifteen-second round, A for thirty-second round, D for sixty-second round

\end{itemize}

\item Problem number

\end{itemize}

For example, 14QII9 would refer to 2014 qualifying stage test II problem 9, 13AI20 would refer to 2013 area stage test I problem 20, 10NA6 would refer to 2010 nationals oral round thirty-second round problem six, while 9N4 would refer to 2009 nationals written round problem 4.

\subsubsection*{Miscellaneous}

\begin{itemize}

\item If you see any typos, errors, broken problems, or if you have any comments, suggestions or corrections, do not hesitate to inform me personally or through \mailto{cjquines0@gmail.com}.

\item Problems are not copied verbatim, sometimes they are copy-edited to reduce space. The problem is, in essence, the same.

\item Please don't be put off by the question just because it is from an area or nationals round. Many area and national oral questions are of similar difficulty to the harder qualifying questions.

\item It's okay if you're absent for a session, just make sure to pick up the problem set afterward. Do not be absent for session 10, however, your presence is very important for that session.

\item Do try to solve all the problems you can. Yes, I know it's hard because there's a lot of schoolwork, but please do try to make an effort -- it takes me time to compile problems as well.

\end{itemize}

\end{document}