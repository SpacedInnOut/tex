\documentclass[11pt,paper=letter]{scrartcl}
\usepackage[utf8]{inputenc}
\usepackage{cjquines}
\usepackage[document]{ragged2e}
\newcommand\ii[2][]{\item[{#1}] \emph{{#2}}}

\begin{document}

\title{Talasalitaang Pangsipnayan}
\author{Carl Joshua Quines}
\date{November 8, 2016}

\maketitle

\begin{abstract}
Ang talasalitaan na ito ay hango sa \emph{Maugnaying Talasalitaang Pang-Agham} na pinamatnugutan ni Gonsalo del Rosario. Ang pangunahing layunin ng talasalitaan nito ay gamitin sa sipnayang paligsahan.

Inaasahan na ang mambabasa ay nagsasalita ng Filipino; maraming salita ay hindi kasama dahil ito ay mga karaniwang ginagamit. Para sa isang mas ganap na listahan, maaaring basahin ang akdang pinaghanguan.

Ang ibang salita ay wala sa sipi, kundi likha ng may-akda. Ang mga salitang ito ay minarkahan ng * matapos ang salita. Para sa puna at mungkahi, maaari akong kausapin sa \mailto{cj@cjquines.com}.
\end{abstract}

\section{Lahatáng katawagán}

\subsection{Larangan}

\begin{multicols}{3}
\begin{description}

\ii[sipnayan]{mathematics}
\ii[bilnuran]{arithmetic}
\ii[panandáan]{algebra}
\ii[palákipanan*]{combinatorics}
\ii[palátangkasan]{set theory}
\ii[sukgisan]{geometry}
\ii[tatsihaan]{trigonometry}
\ii[palábilangan*]{number theory}

\end{description}
\end{multicols}

\subsection{Salitang teknikal}

\begin{multicols}{3}
\begin{description}

\ii[patunay]{proof}
\ii[húnain]{theorem}
\ii[lútasin]{problem}
\ii[kalutasán]{solution}
\ii[katuríngan]{definition}
\ii[halatin]{denote}
\ii[halgá]{value}
\ii[panurò]{index}

\end{description}
\end{multicols}

\section{Bilnuran}

\subsection{Sakilos}

\begin{multicols}{3}
\begin{description}

\ii[magdagdág]{add}
\ii[panagdág]{addend}
\ii[dagup]{sum}
\ii[kabuuán]{total}
\ii[magbawas]{subtract}
\ii[kaibhán]{difference}
\ii[magparami]{multiply}
\ii[bunga]{product}
\ii[maghatì]{divide}
\ii[hátiin]{dividend}
\ii[pahatì]{divisor}
\ii[kahatiàn]{quotient}
\ii[labí]{remainder}

\end{description}
\end{multicols}

\subsection{Bilang}

\begin{multicols}{3}
\begin{description}

\ii[awán]{zero}
\ii[bilang]{number}
\ii[tambilang]{digit}
\ii[tahás]{positive}
\ii[balíng]{negative}
\ii[hatimbilang]{fraction}
\ii[panakdâ]{numerator}
\ii[pamahagi]{denominator}
\ii[baligtád]{reciprocal}
\ii[tagwáy]{ratio}
\ii[takad]{base}
\ii[paulit]{exponent}
\ii[lambál]{power}
\ii[x-dinawak]{x-squared}
\ii[x-binúok]{x-cubed}
\ii[x sa iká-apat na lambál]{x to the fourth power}

\end{description}
\end{multicols}

\section{Panandáan}

\subsection{Bilang}

\begin{multicols}{3}
\begin{description}

\ii[likás]{natural}
\ii[buumbilang]{integer}
\ii[tagwayin]{rational}
\ii[ditagwayin]{irrational}
\ii[tunay]{real}
\ii[guní]{imaginary}
\ii[hugnáy]{complex}
\ii[katápatan]{conjugate}
\ii[hagkís]{amplitude}
\ii[kawagsán]{modulus}
\ii[hangganín]{finite}
\ii[awanggan]{infinity}

\end{description}
\end{multicols}

\subsection{Katumbasan}

\begin{multicols}{3}
\begin{description}

\ii[pahayag]{expression}
\ii[katumbasan]{equation}
\ii[áligin]{variable}
\ii[lágiin]{constant (n.)}
\ii[ilipat]{transpose}
\ii[hulip]{substitute}
\ii[túwirin]{linear}
\ii[dáwakin]{quadratic}
\ii[búukin]{cubic}
\ii[ápatin]{quartic}
\ii[magkalutas]{simultaneous}
\ii[sanyô]{formula}

\end{description}
\end{multicols}

\subsection{Dikatumbasan}

\begin{multicols}{3}
\begin{description}

\ii[dikatumbasan]{inequality}
\ii[tamtaman]{mean}
\ii[bilnuring]{arithmetic}
\ii[sukgising]{geometric}
\ii[súwatuhing]{harmonic}
\ii[dáwaking]{quadratic}
\ii[palípagsasayos*]{rearrangement}
\ii[higdulin]{maximum (adj.)}
\ii[kubdulin]{minimum (adj.)}

\end{description}
\end{multicols}

\subsection{Damikay}

\begin{multicols}{3}
\begin{description}

\ii[damikay]{polynomial}
\ii[antás]{degree}
\ii[takáy]{term}
\ii[katuwáng]{coefficient}
\ii[isákay]{monomial}
\ii[duhakay]{binomial}
\ii[talukay]{trinomial}
\ii[tanda]{sign}
\ii[ugát]{root}
\ii[kabuô]{factor (n.)}
\ii[bungkagin]{factor (v.)}
\ii[talangí]{discriminant}

\end{description}
\end{multicols}

\subsection{Kabisà}

\begin{multicols}{3}
\begin{description}

\ii[kabisà]{function}
\ii[halgáng wagas]{absolute value}
\ii[pariugát]{square root}
\ii[taluugát]{cube root}
\ii[sakláw]{domain}
\ii[kasakláw*]{codomain}
\ii[abot]{range}
\ii[logpaulit]{logarithm}
\ii[takad]{base}
\ii[karaniwan]{common}
\ii[likás]{natural}

\end{description}
\end{multicols}

\subsection{Datig}

\begin{multicols}{3}
\begin{description}

\ii[datig]{sequence}
\ii[bilnuring]{arithmetic}
\ii[sukgising]{geometric}
\ii[panisulong]{progression}
\ii[dalayráy]{series}
\ii[pagdagup]{summation}
\ii[hanggan]{limit}
\ii[palapit]{convergent}
\ii[palayô]{divergent}

\end{description}
\end{multicols}

\subsection{Katayuwat}

\begin{multicols}{3}
\begin{description}

\ii[katayuwat]{coordinates}
\ii[Dekarthíng]{Cartesian}
\ii[múlaan]{origin}
\ii[tuntón]{plot (n.)}
\ii[talangguhit]{graph}
\ii[taluhog]{axis}
\ii[hilig]{slope}
\ii[higawat]{abscissa}
\ii[tayuwat]{ordinate}
\ii[kapatán]{quadrant}
\ii[tugano]{vector}

\end{description}
\end{multicols}

\section{Pálákipanan}

\subsection{Palatangkasan}

\begin{multicols}{3}
\begin{description}

\ii[tangkas]{set}
\ii[walang-laman]{empty}
\ii[mulhagi]{element}
\ii[kapíd]{pair}
\ii[ayós]{order}
\ii[kubtangkas]{subset}
\ii[angkop]{proper}
\ii[kaisahán]{union}
\ii[bagtás]{intersection}
\ii[díugnay]{disjoint}
\ii[lahátan]{universal}
\ii[kahalgá]{equivalent}
\ii[kapunô]{complement}
\ii[sungkád]{bijection}

\end{description}
\end{multicols}

\subsection{Pagpili}

\begin{multicols}{3}
\begin{description}

\ii[pálákipan]{combination}
\ii[pámálitan]{permutation}
\ii[talay]{table}
\ii[hanay]{row}
\ii[tudlíng]{column}
\ii[pasok]{entry}
\ii[búuin]{factorial}
\ii[kalagmitán]{probability}
\ii[saliringing]{independent}
\ii[bigát]{weight}
\ii[alisagâ]{random}
\ii[dagdag-bawas*]{inclusion-exclusion}

\end{description}
\end{multicols}

\section{Sukgísan}

\subsection{Lahátan}

\begin{multicols}{3}
\begin{description}

\ii[hawîg]{similar}
\ii[kalapat]{congruent}
\ii[kaguhit]{collinear}
\ii[katuntong]{concurrent}
\ii[laraw]{figure}
\ii[panihâ]{protractor}
\ii[galod]{ruler}
\ii[parianyuin]{symmetry}
\ii[sabalik]{reflection}
\ii[pariagwat]{equidistant}
\ii[duhati*]{bisector}
\ii[tathati*]{trisector}

\end{description}
\end{multicols}

\subsection{Siksín}

\begin{multicols}{3}
\begin{description}

\ii[siksín]{solid}
\ii[buók]{volume}
\ii[dayag]{surface}
\ii[damdayag]{polyhedron}
\ii[talutô]{cone}
\ii[binuók]{cube}
\ii[tagilô]{pyramid}
\ii[balimbíng]{prism}
\ii[bumbóng]{cylinder}
\ii[timbulog]{sphere}
\ii[habà]{length}
\ii[lapad]{width}
\ii[taas]{height}

\end{description}
\end{multicols}

\subsection{Lapyâ}

\begin{multicols}{3}
\begin{description}

\ii[lapyâ]{plane}
\ii[guhit]{line}
\ii[kilô]{curve}
\ii[putól]{segment}
\ii[tuldok]{point}
\ii[gituldok]{midpoint}
\ii[sinag]{ray}
\ii[tadlóng]{perpendicular}
\ii[agapáy]{parallel}
\ii[bagtâs]{intersection}

\end{description}
\end{multicols}

\subsection{Sihà}

\begin{multicols}{3}
\begin{description}

\ii[sihà]{angle}
\ii[bagtasan]{vertex}
\ii[kipót]{acute}
\ii[tadlóng]{right}
\ii[bikâ]{obtuse}
\ii[ladlád]{straight}
\ii[pabalik]{reflex}
\ii[antás]{degree}
\ii[radyan]{radian}
\ii[kapunô]{complementary}
\ii[kaladlád]{supplementary}

\end{description}
\end{multicols}

\subsection{Damsihà}

\begin{multicols}{3}
\begin{description}

\ii[damsihà]{polygon}
\ii[gilid]{side}
\ii[bagtasan]{vertex}
\ii[dawak]{area}
\ii[gikop]{perimeter}
\ii[hilís]{diagonal}
\ii[tatsihà]{triangle}
\ii[patgilid]{quadrilateral}
\ii[limsihà]{pentagon}
\ii[nimsihà]{hexagon}
\ii[pitsihà]{septagon}
\ii[walsihà]{octagon}
\ii[samsihà]{nonagon}
\ii[pulsihà]{decagon}
\ii[parigilid]{equilateral}
\ii[parisihâ]{equiangular}

\end{description}
\end{multicols}

\subsection{Tatsihà}

\begin{multicols}{3}
\begin{description}

\ii[kaduhain]{isosceles}
\ii[bagtasan]{vertex}
\ii[takad]{base}
\ii[tadlóng]{right}
\ii[tayog]{altitude}
\ii[tadtungâ*]{orthocenter}
\ii[lobtungâ*]{incenter}
\ii[tiktungâ*]{circumcenter}
\ii[gitungâ*]{centroid}

\end{description}
\end{multicols}

\subsection{Patgilid}

\begin{multicols}{3}
\begin{description}

\ii[tagigapáy]{trapezoid}
\ii[parigapáy]{parallelogram}
\ii[tagisukát]{rhombus}
\ii[paritadlóng]{rectangle}
\ii[parisukát]{square}
\ii[tikupan*]{cyclic}
\ii[dikítan*]{tangential}

\end{description}
\end{multicols}

\subsection{Bilog}

\begin{multicols}{3}
\begin{description}

\ii[bilog]{circle}
\ii[hatimbilog]{semicircle}
\ii[tungâ]{center}
\ii[tikop]{circumference}
\ii[bantód]{diameter}
\ii[lihit]{radius}
\ii[bantók]{arc}
\ii[bagtíng]{chord}
\ii[tangab]{secant}
\ii[dikít]{tangent}
\ii[lihà]{sector}
\ii[bumabatak*]{subtend}

\end{description}
\end{multicols}

\subsection{Tatsihaan}

\begin{multicols}{3}
\begin{description}

\ii[sinwáy]{sine}
\ii[kasinwáy]{cosine}
\ii[tanway]{tangent}
\ii[katanway]{cotangent}
\ii[sekwáy]{secant}
\ii[kasekwáy]{cosecant}
\ii[kakabisà]{cofunction}
\ii[batas]{law}

\end{description}
\end{multicols}

\section{Palábilangan}

\begin{multicols}{3}
\begin{description}

\ii[bilang]{number}
\ii[tambilang]{digit}
\ii[tahás]{positive}
\ii[balíng]{negative}
\ii[kaparami]{multiple (n.)}
\ii[pahatì]{divisor}
\ii[kabuô]{factor}
\ii[tukól]{even}
\ii[gansál]{odd}
\ii[lantáy]{prime}
\ii[himpit]{perfect}
\ii[parirami]{square}
\ii[talurami]{cube}

\end{description}
\end{multicols}

\end{document}
